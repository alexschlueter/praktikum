\section{Einführung}
\section{Versuche}
\subsection{Aufgabe 5}
\begin{figure}[H]
  \centering
  \begin{tikzpicture}
    \begin{axis}[
      width=15 cm,
      height=9 cm,
      xmin=0, xmax=30,
      ymin=0, ymax=1.2,
      xlabel={Angelenkte Spannung U [V]},
      ylabel={Gemessene Stromstärke I [A]},
      legend entries={Gleichspannung, Wechselspannung, Linearer Fit beider Spannungen},
      legend pos=north west,
      domain=0.1:28,
    ]
      \addplot plot [only marks,mark=x,thick,error bars/.cd, x dir=both, x fixed=0.25, y dir=both, y fixed=0.05]  file {GleichR.txt};
      \addplot plot [only marks,mark=o,thick,error bars/.cd, x dir=both, x fixed=0.25, y dir=both, y fixed=0.05]  file {WechselR.txt};
      \addplot[mark=none] {0.04*x+0.014};
    \end{axis}
  \end{tikzpicture}
  \caption{Versuch mit ohmschen Widerstand (Spannung gegen Stromstärke)}
  \label{fig:UIOhmscher}
\end{figure}
\begin{table}[H]
  \centering
  \begin{tabular}{c | c | c | c}
    Spannungsart & Variabel m & Variabel b & Varians der Residuals\\ \hline
    Gleichspannung & 0,04 & 0,014 & 0,00024\\
    Wechselspannung & 0,0422 & 0,007 & 0,00077
  \end{tabular}
  \caption{Linearer Fit zu Abbildung \ref{fig:UIOhmscher}}
  \label{tab:fitUIOhmscher}
\end{table}
\begin{figure}[H]
  \centering
  \begin{tikzpicture}
    \begin{axis}[
      width=15 cm,
      height=9 cm,
      xmin=0, xmax=25,
      ymin=0, ymax=30,
      xlabel={Gemessene Leistung P [W]},
      ylabel={Produkt der Stromstärke I und der Spannung U [W]},
      legend entries={Gleichspannung, Wechselspannung, Linearer Fit beider Spannungen},
      legend pos=north west,
      domain=0.1:28,
    ]
      \addplot plot [only marks,mark=x,thick,error bars/.cd, x dir=both, x fixed=0.5, y dir=both, y fixed=0.5]  file {GleichR4.txt};
      \addplot plot [only marks,mark=o,thick,error bars/.cd, x dir=both, x fixed=0.5, y dir=both, y fixed=0.5]  file {WechselR2.txt};
      \addplot[mark=none] {1.004*x+0.25};
    \end{axis}
  \end{tikzpicture}
  \caption{Versuch mit ohmschen Widerstand (Leistung gegen Produkt aus Spannung und Stromstärke)}
  \label{fig:PUIOhmscher}
\end{figure}
\begin{table}[H]
  \centering
  \begin{tabular}{c | c | c | c}
    Spannungsart & Variabel m & Variabel b & Varians der Residuals\\ \hline
    Gleichspannung & 1,005 & 0,352 & 0,004\\
    Wechselspannung & 1,003 & 0,164 & 0,45
  \end{tabular}
  \caption{Linearer Fit zu Abbildung \ref{fig:PUIOhmscher}}
  \label{tab:fitPUIOhmscher}
\end{table}
\subsection{Aufgabe 6}
\begin{figure}[H]
  \centering
  \begin{tikzpicture}
    \begin{axis}[
      width=15 cm,
      height=9 cm,
      xmin=0, xmax=30,
      ymin=0, ymax=1.2,
      xlabel={Angelenkte Spannung U [V]},
      ylabel={Gemessene Stromstärke I [A]},
      legend entries={Wechselspannung, Linearer Fit},
      legend pos=north west,
      domain=0.1:28,
    ]
      \addplot plot [only marks,mark=x,thick,error bars/.cd, x dir=both, x fixed=0.25, y dir=both, y fixed=0.05]  file {WechselL.txt};
      
      \addplot[mark=none] {0.034*x-0.0096};
    \end{axis}
  \end{tikzpicture}
  \caption{Versuch mit Spule (Spannung gegen Stromstärke)}
  \label{fig:UISpule}
\end{figure}
\begin{table}[H]
  \centering
  \begin{tabular}{c | c | c | c}
    Spannungsart & Variabel m & Variabel b & Varians der Residuals\\ \hline
    Wechselspannung & 0,034 & -0,0096 & $2,54\cdot10^{-5}$
  \end{tabular}
  \caption{Linearer Fit zu Abbildung \ref{fig:UISpule}}
  \label{tab:fitUISpule}
\end{table}
\begin{figure}[H]
  \centering
  \begin{tikzpicture}
    \begin{axis}[
      width=15 cm,
      height=9 cm,
      xmin=0, xmax=20,
      ymin=0, ymax=25,
      xlabel={Leistung P [W]},
      ylabel={Produkt der Stromstärke I und der Spannung U [W]},
      legend entries={Wechselspannung, Linearer Fit},
      legend pos=north west,
      domain=0.1:28,
    ]
      \addplot plot [only marks,mark=x,thick,error bars/.cd, x dir=both, x fixed=0.5, y dir=both, y fixed=0.05]  file {WechselL2.txt};
      
      \addplot[mark=none] {1.301*x-0.171};
    \end{axis}
  \end{tikzpicture}
  \caption{Versuch mit Spule (Leistung gegen Produkt aus Spannung und Stromstärke)}
  \label{fig:PUISpule}
\end{figure}
\begin{table}[H]
  \centering
  \begin{tabular}{c | c | c | c}
    Spannungsart & Variabel m & Variabel b & Varians der Residuals\\ \hline
    Wechselspannung & 1,301 & -0,171 & 0,14
  \end{tabular}
  \caption{Linearer Fit zu Abbildung \ref{fig:PUISpule}}
  \label{tab:fitPUISpule}
\end{table}
\subsection{Aufgabe 7}
\begin{figure}[H]
  \centering
  \begin{tikzpicture}
    \begin{axis}[
      width=15 cm,
      height=9 cm,
      xmin=0, xmax=30,
      ymin=0, ymax=1.2,
      xlabel={Angelenkte Spannung U [V]},
      ylabel={Gemessene Stromstärke I [A]},
      legend entries={Gleichspannung, Linearer Fit},
      legend pos=north west,
      domain=0.1:28,
    ]
      \addplot plot [only marks,mark=x,thick,error bars/.cd, x dir=both, x fixed=0.25, y dir=both, y fixed=0.05]  file {GleichL.txt};
      
      \addplot[mark=none] {0.041*x+0.016};
    \end{axis}
  \end{tikzpicture}
  \caption{Versuch mit Spule}
  \label{fig:UIGleichSpule}
\end{figure}
\begin{table}[H]
  \centering
  \begin{tabular}{c | c | c | c}
    Spannungsart & Variabel m & Variabel b & Varians der Residuals\\ \hline
    Wechselspannung &0,041 & 0,016 & 0,00035
  \end{tabular}
  \caption{Linearer Fit zu Abbildung \ref{fig:UIGleichSpule}}
  \label{tab:fitUIGleichSpule}
\end{table}
\subsection{Aufgabe 8}
\begin{figure}[H]
  \centering
  \begin{tikzpicture}
    \begin{axis}[
      width=15 cm,
      height=9 cm,
      xmin=0, xmax=30,
      ymin=0, ymax=0.65,
      xlabel={Angelenkte Spannung U [V]},
      ylabel={Gemessene Stromstärke I [A]},
      legend entries={Wechselspannung, Linearer Fit},
      legend pos=north west,
      domain=0.1:28,
    ]
      \addplot plot [only marks,mark=x,thick,error bars/.cd, x dir=both, x fixed=0.25, y dir=both, y fixed=0.05]  file {WechselLC.txt};
      
      \addplot[mark=none] {0.0229*x+0.00344};
    \end{axis}
  \end{tikzpicture}
  \caption{Versuch mit Spule und Kondensator (Spannung gegen Stromstärke)}
  \label{fig:UILC}
\end{figure}
\begin{table}[H]
  \centering
  \begin{tabular}{c | c | c | c}
    Spannungsart & Variabel m & Variabel b & Varians der Residuals\\ \hline
    Wechselspannung & 0,023 & 0,003 & $1,36\cdot10^{-5}$
  \end{tabular}
  \caption{Linearer Fit zu Abbildung \ref{fig:UILC}}
  \label{tab:fitUILC}
\end{table}
\begin{figure}[H]
  \centering
  \begin{tikzpicture}
    \begin{axis}[
      width=15 cm,
      height=9 cm,
      xmin=0, xmax=10,
      ymin=0, ymax=15,
      xlabel={Leistung P [W]},
      ylabel={Produkt der Stromstärke I und der Spannung U [W]},
      legend entries={Wechselspannung, Linearer Fit},
      legend pos=north west,
      domain=0:9,
    ]
      \addplot plot [only marks,mark=x,thick,error bars/.cd, x dir=both, x fixed=0.5, y dir=both, y fixed=0.05]  file {WechselLC2.txt};
      
      \addplot[mark=none] {1.62*x+0.27};
    \end{axis}
  \end{tikzpicture}
  \caption{Versuch mit Spule(Leistung gegen Produkt aus Spannung und Stromstärke)}
  \label{fig:PUILC}
\end{figure}
\begin{table}[H]
  \centering
  \begin{tabular}{c | c | c | c}
    Spannungsart & Variabel m & Variabel b & Varians der Residuals\\ \hline
    Wechselspannung & 1,62 & 0.27 & 0,038
  \end{tabular}
  \caption{Linearer Fit zu Abbildung \ref{fig:PUILC}}
  \label{tab:fitPUILC}
\end{table}
\section{Diskussion}
\notecite{anleitung-ws2014}
