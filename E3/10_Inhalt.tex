\section{Einführung}
Mit dem Millikanversuch ist es möglich durch Beobachtung der Geschwindigkeit von geladenen Flüssigkeitstropfen in einem Kondensator Rückschlüsse auf die Elementarladung zu ziehen. Als Flüssigkeit wurde ein Öl gewählt, da es sich im Gegensatz zu Wasser kaum verflüchtigt. Diese Eigenschaft ist notwendig, da die Masse der Tropfen während der Beobachtung konstant bleiben muss.

Wenn sich ein Öltropfen zwischen den Kondensatorplatten befindet, wirken die folgenden Kräfte auf ihn:
\begin{table}[H]
	\centering
	\begin{tabular}{r c  c  l}
	Gewichtskraft & $\vec{F}_G$ & = & $m\vec{g}=V\rho_{Öl}\vec{g}=\frac{4}{3}\pi r^3 \rho_{Öl}\vec{g}$\\
	Kraft im elektrischen Feld & $\vec{F}_{el}$ & = & $Q\vec{E}$ mit $|\vec{E}|=\frac{U}{d}$\\
	Auftrieb & $\vec{F}_{A}$ & = & $V\rho_L\vec{g}=\frac{4}{3}\pi r^3 \rho_L\vec{g}$\\
	Reibungskraft & $\vec{F}_{R}$ & = & $6\pi \eta r\vec{v}$
	\end{tabular}
\end{table}
Nach kurzer Zeit spielt sich zwischen diesen Kräften ein Gleichgewicht ein, welches je nach Polung der Kondensatoren und Bewegungsrichtung des Tropfen unterschiedlich ist. Unter der Bedingung, dass sich der Tropfen nicht zu schnell bewegen darf, aber auch nicht fast stehen darf, weil beide Fälle die Messung durch die kurzen zumessenden Zeiten oder den Brownschen Effekt der Molekularbewegung ungenauer machen, ergeben sich folgende Gleichgewichte:
\begin{enumerate}
\item Bei ausgeschaltetem elektrischen Feld:
\begin{equation}
|\vec{F}_G|=|\vec{F}_R|+|\vec{F}_A|
\label{eq:keinfeld}
\end{equation}
\item Bei senkrecht nach unten orientiertem elektrischen Feld und Kraft im elektrischen Feld stärker als die um die Auftriebskraft verminderte Gewichtskraft:
\begin{equation}
|\vec{F}_G|+|\vec{F}_R|=|\vec{F}_{el}|+|\vec{F}_A|
\label{eq:mitfeld}
\end{equation}
\end{enumerate}
Dabei ist es wichtig, dass die beiden Kondensatorplatten waagerecht zueinander sind, da ansonsten die Kraft $\vec{F}_{el}$ nicht parallel zu den anderen Kräften wirken.

Diese beiden Formel \ref{eq:keinfeld} und \ref{eq:mitfeld} können nun nach dem Radius $r$ und der Ladung $Q$ des Tropfen umgestellt werden.(Herleitung siehe Anhang)
\begin{gather}
r=3\sqrt{\frac{\eta v_{\downarrow}}{2(\rho_{Öl}-\rho_L)g}}\label{eq:radius}\\
Q = \frac{18 \pi d}{U}\sqrt{\frac{\eta^3 v_{\downarrow}}{2(\rho_{Öl}-\rho_L)g}}(v_{\downarrow}+v_{\uparrow})\label{eq:ladung}
\end{gather}
\subsection{Korrektur}
Da man Tropfen in einer Größenordnung der freien Weglänge der Luftmoleküle beobachtet, gilt das Stokessche Gesetz nicht bzw. es muss modifiziert werden, da es nur für eine homogene Flüssigkeitsumgebung gilt. So muss mit einer korrigierten Viskosität gerechnet werden.
\begin{align}
\eta_{korr}=\frac{\eta}{1+\frac{A\lambda}{r}}\label{eq:korrektur}
\end{align}
Wobei mit der mittleren freien Weglänge in Luft $\lambda$, dem Tröpfenradius $r$ und einer empirischen Konstante $A=0,863$ gerechnet wird.
\section{Durchführung}
Bei der Zeitbestimmung für die Strecke die ein Tropfen zurücklegt, werden die Zeiten separat für fünf Durchgänge bestimmt und später gemittelt, um eventuelle grobe Messfehler erkennen zu können. 
\section{Auswertung}
Mit \textit{gnuplot} wird nach dem least-squares-Verfahren ein linearer Fit gegen die drei in der Anleitung~\footcite{anleitung-ws2014} angegebenen Werte durchgeführt, um die dynamische Viskosität $\eta$ von Luft bei $T=\SI{21\pm.5}{\degreeCelsius}$ (gemessen mit Thermometer) zu bestimmen.
\begin{figure}[H]
\centering
\begin{tikzpicture}
  \begin{axis}[
    width=15 cm,
    height=9 cm,
    xmin=19, xmax=31,
    ymin=1.815e-5, ymax=1.87e-5,
    xlabel={Temperatur {\si{\degreeCelsius}}},
    ylabel={Dynamische Viskosität von Luft $\eta$ [\si{\pascal.s}]},
    domain=0.1:105,
  ]
  \addplot plot [only marks,mark=o]  coordinates {
    (20,1.819e-5)
    (26.85,1.86e-5)
    (30,1.867e-5)
  };
  \addplot plot [only marks,mark=x, error bars/.cd, x dir=both, x explicit]  coordinates {
    (21,1.82479e-5) +- (.5,0)
  };
  \addplot plot [mark=none] {1.72e-5+x*4.99e-8};
  \end{axis}
\end{tikzpicture}
\caption{Fit der dynamischen Viskosität von Luft}
\label{fig:dynviskos}
\end{figure}
Wir lesen $\eta=\SI{1.824(2)e-5}{\pascal.s}$ ab.

Der Luftdruck wurde mit einem Barometer gemessen: $p=\SI{990.9(1)}{mbar}$.\\

Die Dichte der Luft berechnen wir zu $\rho_L=(T_0\cdot p/(T\cdot p_0))=\SI{1.204}{kg/m^3}$.\\

Für die mittlere freie Weglänge ergibt sich $\lambda=3\eta \cdot \sqrt{\pi \rho_L/(8p)}/\rho_L=\SI{99.32}{nm}$. \\

Wir führen mit \cref{eq:korrektur} pro beobachtetes Teilchen eine Iteration durch, um $\eta_{korr}$ zu bestimmen. Die Werte liegen im Bereich zwischen $\num{.91}\eta$ und $\num{.96}\eta$. \\

Die Dichte von Öl wurde gemittelt aus den in \footcite{anleitung-ws2014} angegebenen Werten: $\rho_{Öl}=\SI{877}{kg/m^3}-(21-15)/(25-15)\cdot \SI{6}{kg/m^3}=\SI{834.4}{kg/m^3}$. \\

Wir berechnen nun die Mittelwerte der Steig- und Fallzeiten, daraus $v_{\uparrow}$ bzw. $v_{\downarrow}$ und daraus mithilfe von \cref{eq:radius} $r$ und mithilfe von \cref{eq:ladung} $Q$.
\begin{sidewaystable}
  \centering
  \begin{tabular}{c c c c c c c c} \toprule
    $U$ [\si{V}] & $t_{\uparrow}$ [\si{s}] & $t_{\downarrow}$ [\si{s}] & $v_{\uparrow}$ [\si{mm/s}] & $v_{\downarrow}$ [\si{mm/s}] & $r$ [\si{\micro m}] & $Q$ [\si{\atto C}] & $Q/e$ \\ \midrule
	\num{578(2)} & \num{10.55(063)} & \num{4.52(027)} & \num{0.0948(00057)} & \num{0.2214(00131)} & \num{1.460(0043)} & \num{1.518(0112)} & \num{9.475(0700)} \\
	\num{575(2)} & \num{18.57(119)} & \num{3.47(015)} & \num{0.0538(00034)} & \num{0.2880(00120)} & \num{1.665(0035)} & \num{1.901(0110)} & \num{11.865(0686)} \\
	\num{571(2)} & \num{23.73(140)} & \num{9.06(027)} & \num{0.0421(00025)} & \num{0.1104(00033)} & \num{1.031(0016)} & \num{0.506(0021)} & \num{3.158(0132)} \\
	\num{571(2)} & \num{21.92(130)} & \num{5.48(032)} & \num{0.0456(00027)} & \num{0.1826(00107)} & \num{1.326(0039)} & \num{0.999(0078)} & \num{6.235(0485)} \\
	\num{569(2)} & \num{18.28(192)} & \num{9.58(036)} & \num{0.0547(00058)} & \num{0.1043(00039)} & \num{1.002(0019)} & \num{0.513(0029)} & \num{3.201(0183)} \\
	\num{500(2)} & \num{11.60(075)} & \num{12.59(015)} & \num{0.0862(00055)} & \num{0.0794(00009)} & \num{0.874(0005)} & \num{0.521(0019)} & \num{3.253(0120)} \\
	\num{500(2)} & \num{12.55(084)} & \num{12.06(022)} & \num{0.0797(00054)} & \num{0.0829(00015)} & \num{0.893(0008)} & \num{0.524(0020)} & \num{3.274(0128)} \\
	\num{564(2)} & \num{11.38(094)} & \num{12.07(027)} & \num{0.0879(00073)} & \num{0.0828(00018)} & \num{0.893(0010)} & \num{0.488(0024)} & \num{3.045(0149)} \\
	\num{564(2)} & \num{14.14(092)} & \num{10.22(055)} & \num{0.0707(00046)} & \num{0.0978(00053)} & \num{0.970(0026)} & \num{0.529(0034)} & \num{3.300(0215)} \\
	\num{563(2)} & \num{12.03(073)} & \num{9.29(034)} & \num{0.0831(00050)} & \num{0.1077(00039)} & \num{1.018(0018)} & \num{0.633(0030)} & \num{3.949(0189)} \\
	\num{564(2)} & \num{24.07(138)} & \num{9.29(010)} & \num{0.0415(00024)} & \num{0.1076(00012)} & \num{1.018(0006)} & \num{0.494(0011)} & \num{3.082(0071)} \\
	\num{563(2)} & \num{15.62(056)} & \num{10.52(060)} & \num{0.0640(00023)} & \num{0.0950(00054)} & \num{0.956(0027)} & \num{0.492(0032)} & \num{3.070(0199)} \\
	\num{563(2)} & \num{7.86(014)} & \num{6.28(023)} & \num{0.1272(00022)} & \num{0.1592(00059)} & \num{1.238(0023)} & \num{1.179(0048)} & \num{7.360(0303)} \\
	\num{564(2)} & \num{14.64(064)} & \num{10.26(026)} & \num{0.0683(00030)} & \num{0.0974(00024)} & \num{0.968(0012)} & \num{0.519(0018)} & \num{3.238(0110)} \\
	\bottomrule
  \end{tabular}
  \caption{Messergebnisse}
  \label{tab:ergebnisse}
\end{sidewaystable}
Wir tragen $Q/e$ für jedes gemessene Teilchen in ein Diagramm ein:

\begin{figure}[H]
\centering
\begin{tikzpicture}
  \begin{axis}[
    width=15 cm,
    height=15 cm,
    xmin=0.1, xmax=14.9,
    ymin=0, ymax=13,
    xlabel={Teilchen Nr.},
    ylabel={$Q/e$},
    domain=0:15,
    ymajorgrids=true
  ]
  \addplot plot [only marks,mark=x,error bars/.cd, y dir=both, y explicit]  coordinates {
    (1,9.475) +- (0,0.7)
    (2,11.865) +- (0,0.686)
    (3,3.158) +- (0,0.132)
    (4,6.235) +- (0,0.485)
    (5,3.201) +- (0,0.183)
    (6,3.253) +- (0,0.120)
    (7,3.274) +- (0,0.128)
    (8,3.045) +- (0,0.149)
    (9,3.300) +- (0,0.215)
    (10,3.949) +- (0,0.189)
    (11,3.082) +- (0,0.071)
    (12,3.070) +- (0,0.199)
    (13,7.360) +- (0,0.303)
    (14,3.238) +- (0,0.110)
  };
  \addplot [mark=none] {3};
  \addplot [mark=none] {4};
  \addplot [mark=none] {6};
  \addplot [mark=none] {7};
  \addplot [mark=none] {9};
  \addplot [mark=none] {10};
  \addplot [mark=none] {12};
  \end{axis}
\end{tikzpicture}
\caption{Vielfache der Elementarladung}
\label{fig:ladungsvielfache}
\end{figure}
\notecite{anleitung-ws2014}

\section{Diskussion}
Zumindest qualitativ ist die Quantifizierbarkeit deutlich zu erkennen. Bei allen Messungen liegt die nächste ganze Zahl entweder in oder nur knapp außerhalb des Fehlerintervalls. Bei Ergebnissen die sehr stark von ganzen Zahlen abweichen ist auch die Unsicherheit groß, so dass von Ausreißern ausgegangen werden kann. Dabei eine mögliche Elementarladung in der Größenordnung $ \SI{1.5e-19}{\meter} $ liegen, da dies der kleinsten Stufengröße zwischen unseren Messwerten entspricht. Jedoch lässt sich dies weder auf alle Ladungen verallgemeinern, noch mit Sicherheit bestätigen, da in unseren Versuchen nur Öltröpfchen und keine anderen Stoffe untersucht wurden, sowie eine relativ große Streuung erkennbar ist. Mit den verfügbaren Geräten war es mitunter schwierig die Messwerte zuverlässig zu erfassen, da oft der beobachtete Tropfen entweder zur Seite oder aus der Fokusebene verschwand. Wären die Geräte zuverlässiger, so könnte man auch langsamere Teilchen betrachten, so dass die Zeit genauer erfasst werden kann und eventuell auch Teilchen mit weniger als $ \SI{.5}{\atto C} $.
