\subsection{Formelherleitung für $r$ und $Q$}
Aus der Formel \ref{eq:keinfeld} ergibt sich nach dem Einsetzen der Kraftformeln
\begin{gather*}
\frac{4}{3}\pi r^3 \rho_{Öl}\vec{g}=\frac{4}{3}\pi r^3 \rho_{L}\vec{g}+6\pi \eta  r \vec{v}_{\downarrow}\\
\iff \frac{4}{3}\pi r^3 (\rho_{Öl}-\rho_{L})\vec{g}=6\pi \eta  r \vec{v}_{\downarrow}\\
\iff r^2=\frac{9\eta \vec{v}_{\downarrow}}{2(\rho_{Öl}-\rho_{L})\vec{g}}
\iff r=\sqrt{\frac{9\eta \vec{v}_{\downarrow}}{2(\rho_{Öl}-\rho_{L})\vec{g}}}
\end{gather*}
Aus der Formel \ref{eq:mitfeld} ergibt sich nach dem Einsetzen der Kraftformeln und der Formel \ref{eq:radius}
\begin{gather*}
\frac{4}{3}\pi r^3 \rho_{Öl}\vec{g}+6\pi \eta  r \vec{v}=Q\frac{U}{d}+\frac{4}{3}\pi r^3 \rho_{L}\vec{g}\\
\iff \frac{4}{3}\pi r^3 (\rho_{Öl}-\rho_{L})\vec{g}+6\pi \eta  r \vec{v}_{\uparrow}=Q\frac{U}{d}\\
\iff Q = \frac{18 \pi d}{U}\sqrt{\frac{\eta^3 v_{\downarrow}}{2(\rho_{Öl}-\rho_L)g}}(v_{\downarrow}+v_{\uparrow})
\end{gather*}
\subsection{Zeichnung zu den Kräftegleichgewichten}
\begin{figure}[h]
  \centering
  \includegraphics[width=1\textwidth]{kraftegleichgewichte.png}
  \caption{Kräftegleichgewichte}
  \label{fig:kraftegleich}
\end{figure}