\section{Einführung}
Resonanzkreise werden im allgemeinen als Frequenzfilter genutzt, um das Empfangen von nur speziellen Frequenzen zu ermöglichen.
\subsection{Serienresonanzkreis}
\begin{figure}[H]
	\centering
	\includegraphics[width=.4\textwidth]{elresonanz_reihe}
	\caption{Schaltbild des Serienresonanzkreises}
	\label{fig:aufbaureihe}
\end{figure}
Bei dem Serienresonanzkreis sind Ohmsche Widerstände, Spulen und Kondensatoren in Reihe geschaltet. Bei der Berechnung werden dazu alle Ohmschen Widerstände und Innenwiderstände zu einem Gesamtwiderstand zusammengefasst und analog auch die Kondensatoren und Spulen zu Gesamtkapazitäten und Gesamtinduktivitäten zusammengefasst. Im einfachsten Fall ergibt sich das Schaltbild Abbildung \ref{fig:aufbaureihe}. Diese Schaltung wird mit der komplexen Betrachtung des Wechselstroms berechnet. Die komplexe Impendanz setzt sich bei Reihenschaltung additiv zusammen. Es gilt \begin{equation}
	Z_G = Z_R + Z_L + Z_C = R + \mathrm i\omega L + \frac{1}{\mathrm{i}\omega C} = R + \mathrm i \left(\omega L - \frac{1}{\omega L}\right)
\end{equation}
Außerdem gilt analog zu Ohmschen Gesetz im Wechselstromkreis
\begin{equation}
	|I| = \frac{|U|}{|Z_G|} = \frac{|U|}{\sqrt{R^2 + \left(\omega L - \frac{1}{\omega C}\right)^2}} \label{eq:serienstrom}
\end{equation}
Die Resonanz ist bei maximaler Stromstärke erreicht. Da der Ohmsche Widerstand und die Spannung bezüglich der Frequenz konstant sind ist dies erreicht für
\begin{equation}
	\omega_0 L - \frac{1}{\omega_0 C} = 0 \quad \Leftrightarrow \quad \omega_0 = \frac{1}{\sqrt{LC}} \label{eq:resbed}
\end{equation}
und es gilt 
\begin{equation}
	|I_{max}| = |I(\omega_0)| = \frac{|U|}{R}
\end{equation}
Kennt man nun die Kapazitäten $ C_1, C_2 $, für die $ |I| = \frac{|I_{max}|}{\sqrt{2}} $ gilt kann man daraus den Ohmschen Widerstand errechnen. In diesem Fall gilt \begin{align}
	|Z| &= \sqrt 2 |Z_{min}| = \sqrt{2 R^2} \stackrel{!}{=} \sqrt{R^2 + \left(\omega_0 L - \frac{1}{\omega_0 C_{1,2}}\right)^2} \\
	\Rightarrow~ R^2 &= \left(\omega_0 L - \frac{1}{\omega_0 C{1,2}}\right)^2 = \left(\frac{1}{\omega_0C_0} - \frac{1}{\omega_0 C{1,2}}\right)^2 \\
	\Rightarrow R &= \frac{1}{\omega_0} \left|\frac{1}{C_0} - \frac{1}{C_{1,2}}\right|
		= \frac{1}{2\omega_0} \left(\left|\frac{1}{C_0} - \frac{1}{C_1}\right| + \left|\frac{1}{C_0} - \frac{1}{C_2}\right|\right) \\
		&= \frac{1}{2\omega_0} \left|\left(\frac{1}{C_0} - \frac{1}{C_1}\right) - \left(\frac{1}{C_0} - \frac{1}{C_2}\right)\right| 
		= \frac{1}{2\omega_0} \left|\frac{1}{C_2} - \frac{1}{C_1}\right|
\end{align}
\subsection{Parallelresonanzkreis}
Beim Parallelresonanzkreis werden analog zum Serienresonanzkreis ohmscher Widerstand, Kondensator und Spule parallel geschaltet. Daher müssen zur Bestimmung der Gesamtimpendanz die Kehrwerte der Impendanzen addiert werden \begin{equation}
	\frac{1}{|Z|} = |Y| = \left|\frac{1}{R} + \frac{1}{\mathrm i \omega L} + \mathrm i \omega C\right| = \sqrt{\frac{1}{R^2} + \left(\omega C - \frac{1}{\omega L}\right)^2}
\end{equation}
Zudem gilt \begin{equation}
	|I| = \frac{|U|}{|Z|} = |Y||U| = \sqrt{\frac{1}{R^2} + \left(\omega C - \frac{1}{\omega L}\right)^2} U
\end{equation}
Analog zur Serienresonanzschaltung wir $ |Y_{min}| $ erreicht für \begin{equation}
	\omega C - \frac{1}{\omega L} = 0 \quad \Leftrightarrow \quad \omega = \frac{1}{\sqrt{LC}}
\end{equation}
\subsection{Güte}
Damit eine Schaltung möglichst gut als Frequenzfilter genutzt werden kann, ist es wichtig, dass der Peak bei dieser Frequenz möglichst schmal ist. Daher bezeichnet man als Güte $ Q $ die Überhöhung des Resonanzpeaks. \begin{equation}
	|U_{L,max}| = |U_{C,max}| = Q|U| \qquad \text{nach Definition von Q}
\end{equation}
Da Höhe und Breite des Peaks im direkten Zusammenhang stehen, lässt sich die Güte auch durch die Halbwertsbreite der Resonanzkurve bestimmen. Es gilt
\begin{equation}
	Q = \frac{\omega_0}{\Delta \omega}
\end{equation}
\section{Versuch: Serienresonanzkreis}
Wir überprüfen den aus \cref{eq:serienstrom} erwarteten Zusammenhang, indem wir die obigen Daten mit gnuplot gegen die Funktion $I(C)$ fitten. Wir erhalten:
\begin{table}[H]
  \centering
  \begin{tabular}{c c c c} \toprule
    Parameter & für \SI{500}{\ohm} & für \SI{200}{\ohm} & für \SI{0}{\ohm} \\ \midrule
    $R$ & \SI{694(326)}{\ohm} & \SI{385(283)}{\ohm}  & \SI{30(2041)}{\ohm} \\
    $L$ & \SI{94.06(53)}{\henry} & \SI{93.88(24)}{\henry} & \SI{93.84(16)}{\henry} \\ \bottomrule 
  \end{tabular}
  \caption{Fit}
  \label{tab:serienfit}
\end{table}
Es fällt eine gute Übereinstimmung in den Werten für $L$ auf. Dagegen haben die $R$-Werte jeweils einen so hohen Fehler, dass sie fast unbrauchbar sind.
\begin{figure}[H]
\centering
\begin{tikzpicture}
  \begin{axis}[
    width=15 cm,
    height=9 cm,
    xmin=80, xmax=105,
    ymin=0, ymax=7,
    xlabel={$1/C$ [\si{mF^{-1}}]},
    ylabel={$I$ [\si{mA}]},
    domain=50:250,
    cycle list name=color list,
    legend entries={$500 \si{\ohm}$, $200 \si{\ohm}$, $0 \si{\ohm}$, Fit für $500 \si{\ohm}$, Fit für $200 \si{\ohm}$, Fit für $0 \si{\ohm}$}
  ]
  \addplot+ plot [only marks,mark=x]  table[x expr=1/\thisrowno{0}, y index=2] {serie.csv};
  \addplot+ plot [only marks,mark=asterisk]  table[x expr=1/\thisrowno{3}, y index=5] {serie.csv};
  \addplot+ plot [only marks,mark=square]  table[x expr=1/\thisrowno{6}, y index=8] {serie.csv};
  \pgfplotsset{cycle list shift=-3};
  \addplot+ plot [samples=500, mark=none] {1000*2.01/sqrt(694^2+(1000*94.06-x/(1000*10^(-6)))^2)};
  \addplot+ plot [samples=500, mark=none] {1000*2.01/sqrt(385^2+(1000*93.88-x/(1000*10^(-6)))^2)};
  \addplot+ plot [samples=500, mark=none] {1000*2.01/sqrt(30^2+(1000*93.84-x/(1000*10^(-6)))^2)};
  \end{axis}
\end{tikzpicture}
\caption{Fit der Stromstärke in Abhängigkeit des Kehrwertes der Kapazität}
\label{fig:serienfit}
\end{figure}
Obwohl die Werte direkt gegen den aus der Theorie erwarteten Zusammenhang \cref{eq:serienstrom} gefittet wurden, ist ein großer Unterschied zwischen Fit und Messwerten erkennbar. Der Ort des Maximums wird vom Fit noch recht gut getroffen, jedoch weichen die Höhe des Maximums sowie der Abfall stark ab: Die Messwerte fallen deutlich langsamer als der Fit zu beiden Seiten des Maximums. Auch die aus der Theorie erwartete Symmetrie ist bei unseren Messwerten nicht gegeben. Wir nehmen an, dass die regelbare Kapazität zu stark Fehleranfällig war. Auch während der Durchführung ist Aufgefallen, das oft nicht der eingestellte Wert dem Tatsächlichen entsprach, und sich dies erst besserte wenn man den Regler festhielt um das Einrasten zu erzwingen.

\subsection{Innenwiderstand der Spule}
Mit einem Multimeter lies sich ein Widerstand von $ \SI{52.2(1)}{\ohm} $ bestimmen. Jedoch ist kein Vergleich mit den Ergebnissen der vorherigen Versuchen möglich, da dort eine zu große Unsicherheit angenommen werden musste. Zumindest stimmte aber die Größenordnung der vorherigen Ergebnisse.
\section{Versuch: Parallelresonanzkreis}
An dem Aufbau \ref{fig:aufbauparallel} wird eine Spannung von $U_\approx=\pm 5V$ mit einer Frequenz von $F=(1 \pm 0,005) kHz$ angelegt und es wird der Spannungsabfall am $10\Omega$ Widerstand für unterschiedliche Kapazitäten und Widerstände $R_p$ bestimmt. Der Fehler der Kapazität war mit $1\% $ der eingestellten Kapazität gegeben und der Fehler der Spannung ergab sich aus der Genauigkeit des Messgeräts.
\begin{figure}[h]
  \centering
  \includegraphics[width=.9\textwidth]{Bauplanparall.png}
  \caption{Aufbau eines Parallelresonanzkreises}
  \label{fig:aufbauparallel}
\end{figure}
 Die Stromstärke konnte nun nach der Formel $I=\frac{U}{R}$ berechnet werden. Mit der gleichen Formel wurde mit Hilfe der Gauß'sches Fehlerfortpflanzung der Fehler der Stromstärke bestimmt (Siehe Anhang).
\subsection{$10 k\Omega$ Widerstand}
\begin{table}[h]
  \centering
  
    \begin{tabular}{rrrrrr}
    \toprule
    Kapazität C [µF] & Fehler [µF] & Spannung U [mV] & Fehler [mV] & Stromstärke I [mA] & Fehler [mA] \\
    \midrule
    
        0,000 & 0,000 & 62,900 & 0,100 & 6,290 & 0,315 \\
        0,050 & 0,001 & 52,100 & 0,100 & 5,210 & 0,261 \\
        0,100 & 0,001 & 41,500 & 0,100 & 4,150 & 0,208 \\
        0,147 & 0,001 & 31,500 & 0,100 & 3,150 & 0,158 \\
        0,150 & 0,002 & 30,800 & 0,100 & 3,080 & 0,154 \\
        0,180 & 0,002 & 24,500 & 0,100 & 2,450 & 0,123 \\
        0,210 & 0,002 & 18,300 & 0,100 & 1,830 & 0,092 \\
        0,240 & 0,002 & 12,600 & 0,100 & 1,260 & 0,064 \\
        0,253 & 0,003 & 10,500 & 0,100 & 1,050 & 0,053 \\
        0,270 & 0,003 & 8,400 & 0,100 & 0,840 & 0,043 \\
        0,289 & 0,003 & 7,400 & 0,100 & 0,740 & 0,038 \\
        0,300 & 0,003 & 7,900 & 0,100 & 0,790 & 0,041 \\
        0,321 & 0,003 & 10,500 & 0,100 & 1,050 & 0,053 \\
        0,330 & 0,003 & 12,000 & 0,100 & 1,200 & 0,061 \\
        0,360 & 0,004 & 17,800 & 0,100 & 1,780 & 0,090 \\
        0,390 & 0,004 & 24,000 & 0,100 & 2,400 & 0,120 \\
        0,420 & 0,004 & 30,500 & 0,100 & 3,050 & 0,153 \\
        0,425 & 0,004 & 31,500 & 0,100 & 3,150 & 0,158 \\
        0,450 & 0,005 & 37,000 & 0,100 & 3,700 & 0,185 \\
        0,500 & 0,005 & 47,600 & 0,100 & 4,760 & 0,238 \\
        0,900 & 0,009 & 132,200 & 0,100 & 13,220 & 0,661 \\
    
    \bottomrule
    \end{tabular}
  \caption{Messwerte mit einem $10 k\Omega$ Widerstand}
  \label{tab:10kparallel}
\end{table}

Im folgenden wurde die Werte von C und I im Diagramm aufgetragen.
\begin{figure}[h]
  \centering
  \includegraphics[width=.9\textwidth]{10k.png}
  \caption{$10k\Omega$ Widerstand}
  \label{fig:10k}
\end{figure}

Dem anscheinend polynomischen Verlaufs nach wurden die Messwerte beider Prozesse zusammen gegen die Funktion $I(C)=a\cdot x^6+b\cdot x^5+c\cdot x^4+d\cdot x^3+e\cdot x^2+f\cdot x+g$ mit \emph{gnuplot} nach dem \emph{least-squares}-Verfahren gefittet.
\begin{table}[H]
  \centering
  \begin{tabular}{c | c | c }
    Variabel & Wert & Fehler\\ \hline
    a  & 14329,9 & $\pm 4004$\\
    b & -24283,7 & $\pm6033$\\
    c  & 14789,3 & $\pm 3449$\\
    d  & -3881,36 & $\pm 930$\\
    e  & 429,052 & $\pm 118,6$\\
    f  & -37,299 & $\pm 6,064$\\
    g  & 6,311 & $\pm 0,088$\\
  \end{tabular}
  \caption{Linearer Fit zu Abbildung \ref{fig:10k}}
  \label{tab:fit10k}
\end{table}

Daraus ergibt sich ein Minimum von
\begin{equation}
I_{min}=(0,702\pm 0,035)mA \Rightarrow I_{min}\cdot \sqrt{2}=(0,993\pm0,053)mA
\end{equation}
mit $C_{min}=0,285 \mu F$ und $C_{1}=0,257 \mu F$ und $C_{2}=0,314 \mu F$ mit jeweils $1\%$ Fehler.

Für die Spule lässt sich nun sagen, dass
\begin{equation}
L=\frac{1}{(2\pi f)^2\cdot C}=88,88mH
\end{equation}
\begin{equation}
\Delta L =\sqrt{(\frac{\Delta f}{2\pi^2 f^3 C})^2+(\frac{\Delta C}{2\pi^2 f^2 C^2})^2}=1,24mH
\end{equation}
So ergibt sich für die Spule $L=(88,88 \pm 1,24)mH$.

Der Verlustwiderstand der Schaltung ergibt sich aus
\begin{equation}
R_1= \frac{1}{2\pi f (C_2-C_1)}=2792,192\Omega
\end{equation}
\begin{equation}
\Delta R_1 = \sqrt{ \left( \dfrac{\Delta f}{\pi f^2 (C_2 - C_1)} \right)^2 + \left( \dfrac{\Delta C_2}{\pi f (C_2 - C_1)} \right)^2 + \left( \dfrac{\Delta C_1}{\pi f (C_2 - C_1)} \right)^2}=27,82\Omega
\end{equation}

\newpage
\subsection{$\infty$ Widerstand}

\begin{table}[h]
  \centering
  
    \begin{tabular}{rrrrrr}
    \toprule
    Kapazität C [µF] & Fehler [µF] & Spannung U [mV] & Fehler [mV] & Stromstärke I [mA] & Fehler [mA] \\
    \midrule
    0,000 & 0,000 & 63,000 & 0,100 & 6,300 & 0,330 \\
        0,080 & 0,001 & 45,600 & 0,100 & 4,560 & 0,249 \\
        0,110 & 0,001 & 39,200 & 0,100 & 3,920 & 0,220 \\
        0,140 & 0,001 & 32,600 & 0,100 & 3,260 & 0,191 \\
        0,170 & 0,002 & 26,000 & 0,100 & 2,600 & 0,164 \\
        0,200 & 0,002 & 19,500 & 0,100 & 1,950 & 0,140 \\
        0,230 & 0,002 & 13,100 & 0,100 & 1,310 & 0,120 \\
        0,260 & 0,003 & 7,100 & 0,100 & 0,710 & 0,106 \\
        0,269 & 0,003 & 5,600 & 0,100 & 0,560 & 0,104 \\
        0,289 & 0,003 & 4,000 & 0,100 & 0,400 & 0,102 \\
        0,305 & 0,003 & 5,600 & 0,100 & 0,560 & 0,104 \\
        0,320 & 0,003 & 8,300 & 0,100 & 0,830 & 0,108 \\
        0,350 & 0,004 & 14,600 & 0,100 & 1,460 & 0,124 \\
        0,380 & 0,004 & 21,100 & 0,100 & 2,110 & 0,145 \\
        0,410 & 0,004 & 27,900 & 0,100 & 2,790 & 0,172 \\
        0,440 & 0,004 & 34,400 & 0,100 & 3,440 & 0,199 \\
        0,470 & 0,005 & 41,000 & 0,100 & 4,100 & 0,228 \\
        0,500 & 0,005 & 47,400 & 0,100 & 4,740 & 0,257 \\
        \bottomrule
    
    \end{tabular}
  \label{tab:addlabel}
  \caption{Messwerte mit einem unendlichem Widerstand}
\end{table}

Im folgenden wurde die Werte von C und I im Diagramm aufgetragen.
\begin{figure}[h]
  \centering
  \includegraphics[width=.9\textwidth]{unendlich.png}
  \caption{$\infty\Omega$ Widerstand}
  \label{fig:unendlich}
\end{figure}

Dem anscheinend polynomischen Verlaufs nach wurden die Messwerte beider Prozesse zusammen gegen die Funktion $I(C)=a\cdot x^6+b\cdot x^5+c\cdot x^4+d\cdot x^3+e\cdot x^2+f\cdot x+g$ mit \emph{gnuplot} nach dem \emph{least-squares}-Verfahren gefittet.
\begin{table}[H]
  \centering
  \begin{tabular}{c | c | c }
    Variabel & Wert & Fehler\\ \hline
    a  & 27020,7 & $\pm 6518$\\
    b & -44927,8 & $\pm10090$\\
    c  & 27579,5 & $\pm 5963$\\
    d  & -7512,98 & $\pm 1669$\\
    e  & 901,99 & $\pm 219,9$\\
    f  & -59,7594 & $\pm 11,03$\\
    g  & 6,309 & $\pm 0,122$\\
  \end{tabular}
  \caption{Linearer Fit zu Abbildung \ref{fig:unendlich}}
  \label{tab:fitinfi}
\end{table}

Daraus ergibt sich ein Minimum von
\begin{equation}
I_{min}=(0,580\pm 0,037)mA \Rightarrow I_{min}\cdot \sqrt{2}=(0,820\pm0,047)mA
\end{equation}
mit $C_{min}=0,285 \mu F$ und $C_{1}=0,25 \mu F$ und $C_{2}=0,32 \mu F$ mit jeweils $1\%$ Fehler.

Für die Spule lässt sich nun sagen, dass
\begin{equation}
L=\frac{1}{(2\pi f)^2\cdot C}=88,88mH
\end{equation}
\begin{equation}
\Delta L =\sqrt{(\frac{\Delta f}{2\pi^2 f^3 C})^2+(\frac{\Delta C}{2\pi^2 f^2 C^2})^2}=1,24mH
\end{equation}
So ergibt sich für die Spule $L=(88,88 \pm 1,24)mH$.

Der Verlustwiderstand der Schaltung ergibt sich aus
\begin{equation}
R_1= \frac{1}{2\pi f (C_2-C_1)}=2273,642\Omega
\end{equation}
\begin{equation}
\Delta R_1 = \sqrt{ \left( \dfrac{\Delta f}{\pi f^2 (C_2 - C_1)} \right)^2 + \left( \dfrac{\Delta C_2}{\pi f (C_2 - C_1)} \right)^2 + \left( \dfrac{\Delta C_1}{\pi f (C_2 - C_1)} \right)^2}=29,00\Omega
\end{equation}

\subsection{$2k\Omega$ Widerstand}
\begin{table}[]
  \centering
 
    \begin{tabular}{rrrrrr}
    \toprule
    Kapazität C [µF] & Fehler [µF] & Spannung U [mV] & Fehler [mV] & Stromstärke I [mA] & Fehler [mA] \\
    \midrule
    0,000 & 0,000 & 64,200 & 0,100 & 6,420 & 0,336 \\
    0,080 & 0,001 & 48,200 & 0,100 & 4,820 & 0,261 \\
    0,110 & 0,001 & 42,500 & 0,100 & 4,250 & 0,235 \\
    0,140 & 0,001 & 37,000 & 0,100 & 3,700 & 0,210 \\
    0,170 & 0,002 & 31,600 & 0,100 & 3,160 & 0,187 \\
    0,198 & 0,002 & 27,100 & 0,100 & 2,710 & 0,168 \\
    0,200 & 0,002 & 26,800 & 0,100 & 2,680 & 0,167 \\
    0,230 & 0,002 & 22,800 & 0,100 & 2,280 & 0,152 \\
    0,260 & 0,003 & 20,300 & 0,100 & 2,030 & 0,142 \\
    0,289 & 0,003 & 19,200 & 0,100 & 1,920 & 0,139 \\
    0,320 & 0,003 & 20,500 & 0,100 & 2,050 & 0,143 \\
    0,350 & 0,004 & 23,600 & 0,100 & 2,360 & 0,155 \\
    0,375 & 0,004 & 27,100 & 0,100 & 2,710 & 0,168 \\
    0,380 & 0,004 & 27,800 & 0,100 & 2,780 & 0,171 \\
    0,410 & 0,004 & 33,000 & 0,100 & 3,300 & 0,193 \\
    0,440 & 0,004 & 38,400 & 0,100 & 3,840 & 0,216 \\
    0,470 & 0,005 & 44,200 & 0,100 & 4,420 & 0,243 \\
    0,500 & 0,005 & 49,800 & 0,100 & 4,980 & 0,268 \\
    \bottomrule
    \end{tabular}
  \label{tab:addlabel}
   \caption{Messwerte mit einem $2 k\Omega$ Widerstand}
\end{table}
Im folgenden wurde die Werte von C und I im Diagramm aufgetragen.
\begin{figure}[]
  \centering
  \includegraphics[width=.9\textwidth]{2k.png}
  \caption{$2k\Omega$ Widerstand}
  \label{fig:2k}
\end{figure}

Dem anscheinend polynomischen Verlaufs nach wurden die Messwerte beider Prozesse zusammen gegen die Funktion $I(C)=a\cdot x^6+b\cdot x^5+c\cdot x^4+d\cdot x^3+e\cdot x^2+f\cdot x+g$ mit \emph{gnuplot} nach dem \emph{least-squares}-Verfahren gefittet.
\begin{table}[H]
  \centering
  \begin{tabular}{c | c | c }
    Variabel & Wert & Fehler\\ \hline
    a  & 6919,65 & $\pm 909,9$\\
    b & -11829,1 & $\pm1414$\\
    c  & 7273,85 & $\pm 838,5$\\
    d  & -1880,89 & $\pm235,2$\\
    e  & 220,707& $\pm 30,92$\\
    f  & -29,17 & $\pm 1,547$\\
    g  & 6,42 & $\pm 0,017$\\
  \end{tabular}
  \caption{Linearer Fit zu Abbildung \ref{fig:2k}}
  \label{tab:fit2k}
\end{table}

Daraus ergibt sich ein Minimum von
\begin{equation}
I_{min}=(1,95\pm 0,140)mA \Rightarrow I_{min}\cdot \sqrt{2}=(2,7577\pm0,160)mA
\end{equation}
mit $C_{min}=0,285 \mu F$ und $C_{1}=0,194 \mu F$ und $C_{2}=0,38 \mu F$ mit jeweils $1\%$ Fehler.

Für die Spule lässt sich nun sagen, dass
\begin{equation}
L=\frac{1}{(2\pi f)^2\cdot C}=88,88mH
\end{equation}
\begin{equation}
\Delta L =\sqrt{(\frac{\Delta f}{2\pi^2 f^3 C})^2+(\frac{\Delta C}{2\pi^2 f^2 C^2})^2}=1,24mH
\end{equation}
So ergibt sich für die Spule $L=(88,88 \pm 1,24)mH$.

Der Verlustwiderstand der Schaltung ergibt sich aus
\begin{equation}
R_1= \frac{1}{2\pi f (C_2-C_1)}=855,67\Omega
\end{equation}
\begin{equation}
\Delta R_1 = \sqrt{ \left( \dfrac{\Delta f}{\pi f^2 (C_2 - C_1)} \right)^2 + \left( \dfrac{\Delta C_2}{\pi f (C_2 - C_1)} \right)^2 + \left( \dfrac{\Delta C_1}{\pi f (C_2 - C_1)} \right)^2}=8,56\Omega
\end{equation}

\subsection{Innenwiderstand der Spule}

Bei der direkten Bestimmung des Innenwiderstands der Spule mit Hilfe des Multimeters ergab sich
\begin{equation}
R_{innen}=(18,9\pm0,1)\Omega
\end{equation}

Bei der Bestimmung aus den Resonanzkurven nutzt man den Umstand, dass bei $R_p = \infty$ gilt

\begin{equation}
R_i=\frac{(2\pi f)^2L^2}{R}=31,186
\end{equation}
\section{Diskussion}
\subsection{Parallelresonanzkreis}
Die Werte für die Induktivität der Spule stimmten bei allen Messungen überein, es gab nur Unterschiede außerhalb des Messgenauigkeit. Die Werte für den Innenwiderstand der Spule hingegen weichen deutlich von einander ab. Dies liegt nicht mehr im Rahmen der Messungenauigkeiten und ist auf eine Erwärmung der Spule oder andere Einflüsse zurück zuführen. 
