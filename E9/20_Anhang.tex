\subsection{Fehlerrechnung}
\subsubsection{Magnetfeld aus Induktionsspannung}
Da Windungszahl und Fläche der Spule sowie die Frequenz des Wechselstroms als nicht Fehlerbehaftet angesehen werden, muss nur die Unsicherheit bei der Spannungsmessung berücksichtigt werden. Es gilt 
\begin{equation}
	\Delta B_{eff} = \left|\frac{\partial }{\partial U_{eff}} \frac{U_{eff}}{\omega NF} \Delta U_{eff}\right| = \left|\frac{\Delta U_{eff}}{\omega N F}\right|
\end{equation}

\subsection{Volumen Zylinder}
Das Volumen eines Zylinders ist gegeben durch \begin{equation*}
	V = \pi h R^2
\end{equation*}
Nach gauß'scher Fehlerfortpflanzung ist die Unsicherheit gegeben durch
\begin{align}
	\Delta V &= \pi \sqrt{\left(\frac{\partial}{\partial h} h R^2 \Delta h\right)^2 + \left( \frac{\partial}{\partial R} hR^2 \Delta R\right)^2} \notag \\
	&= \pi \sqrt{\left(R^2 \Delta h\right)^2 + \left(  2hR \Delta R\right)^2} \label{eq:zylinderErr}
\end{align}