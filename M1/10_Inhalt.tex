\section{Einführung}
\subsection{Federpendel}
Ein \textbf{Federpendel} besteht aus einer Feder mit Federkonstante $D$, an der eine Waagschale mit einer aufgelegten Masse $m$ hängt. Wird diese aus der Ruheposition um $x$ ausgelenkt, so wirkt eine Rückstellkraft $F$ der Auslenkung entgegen. Nach dem \textbf{Hookeschen Gesetz} ist der Zusammenhang linear:
\begin{equation}
  m\ddot{x}=F=-D x
  \label{eq:hook}
\end{equation}
Da die Feder mit der Waagschale eine Eigenmasse $m_F$ besitzt, ist folgender Ansatz realistischer:
\begin{equation}
  (m+m_F/3)\ddot{x}=-D x
  \label{eq:hookreal}
\end{equation}
Mit den Anfangsbedingungen $x(0)=x_0$, $\dot{x}(0)=0$ lautet die Lösung der Bewegungsgleichung
\begin{equation}
  x(t)=x_0 \cos\left(\sqrt{\frac{D}{m+m_F/3}}\cdot t\right).
  \label{eq:federloesung}
\end{equation}
Die Feder führt also harmonische Schwingungen aus. Aus der Periodendauer $T$ kann die Federkonstante bestimmt werden:
\begin{equation}
  D=\frac{4\pi^2}{T^2}\left(m+\frac{m_F}{3}\right)
  \label{eq:federkonst}
\end{equation}
\subsection{Mathematisches Pendel}
Ein \textbf{Mathematisches Pendel} ist ein Fadenpendel der Länge $l$, wobei angenommen wird, dass der Faden masselos ist, die gesamte Masse im Schwerpunkt vereinigt ist und die Bewegung reibungsfrei abläuft. Als Rückstellkraft bei einer Auslenkung um einen Winkel $\varphi$ wirkt die Schwerkraft $F_\varphi=-mg\sin\varphi\approx -mg\varphi$. Die Bewegungsgleichung und Lösung lauten dann:
\begin{align}
  &\ddot{\varphi}+\frac{g}{l}\varphi=0, \qquad \varphi(0)=\varphi_0, \qquad \dot{\varphi}(0)=0 \\
  &\varphi(t)=\varphi_0 \cos\left(\sqrt{\frac{g}{l}}t\right)
  \label{eq:pendelbwgl}
\end{align}
Die Erdbeschleunigung $g$ lässt sich aus der Periodendauer $T$ einer Schwingung ermitteln:
\begin{equation}
  g=\frac{4\pi^2l}{T^2}
  \label{eq:erdbeschl}
\end{equation}
\subsection{Gekoppeltes Pendel}
Bei der Betrachtung des gekoppelten Pendels nimmt man zunächst einige Voraussetzungen an
\begin{enumerate}
\item Beide Pendel schwingen in einer Ebene
\item Beide Pendel haben die gleiche Länge $L$, die gleiche Masse $m$ und die gleiche Eigenfrequenz
\item Der Auslenkungswinkel $\phi$ bleibt im Laufe des Versuchs klein, weshalb man  sagen kann $sin \phi \approx \phi$
\item Als Maß der Auslenkung wird der horizontaler Abstand $x$ des Pendels angenommen
\end{enumerate}

Unter diesen Annahmen kann man nun für das gekoppelte Pendel folgende Bewegungsgleichungen bestimmen.
\begin{gather}
P_1: m\ddot{x}_1= -D_0x_1-D_F(x_1-x_2)\\
P_2: m\ddot{x}_2= -D_0x_2+D_F(x_1-x_2)
\end{gather}
Diese gekoppelten Gleichungen lassen sich mit Hilfe der Substitutionen $z_1=x_1-x_2$ und $z_2=x_1+x_2$ entkoppeln, und erhält die unabhängigen Gleichungen
\begin{gather}
\label{ent1}\ddot{z}_1+(\omega_0^2+\frac{2D_F}{m}z_1)=0\\
\label{ent2}\ddot{z}_2+\omega_0^2z_2 = 0.
\end{gather}
Nun betrachtet man zwei Spezialfälle der Bewegung des gekoppelten Pendels
\begin{enumerate}
\item Gleichsinnige Bewegung: $x_1=x_2$ und damit $z_1=0$. So ergibt sich aus \ref{ent2}
\begin{equation}
\omega_{gl}=\omega_0
\end{equation}
\item Gegensinnige Bewegung: $x_1=-x_2$ und damit $z_2=0$. So ergibt sich aus \ref{ent1}
\begin{equation}
\omega_{geg}=\omega_0\sqrt{1+2D_F/D_0}
\end{equation}
\end{enumerate}
Wenn man nun \ref{ent1} und \ref{ent2} mit den Anfangsbedingungen $x_1(0)=x_0$, $x_2(0)=0$ und $\dot{x}_1(0)=\dot{x}_2(0)=0$ löst, erhält man
\begin{gather}
x_1=x_0 cos[\frac{1}{2}(\omega_{geg}-\omega{gl})t]\cdot cos[\frac{1}{2}(\omega_{geg}+\omega{gl})t]\\
x_2=x_0 sin[\frac{1}{2}(\omega_{geg}-\omega{gl})t]\cdot sin[\frac{1}{2}(\omega_{geg}+\omega{gl})t].
\end{gather}
Weiter gilt
\begin{gather}
\label{eq:tschnell}T=\frac{4\pi}{(\omega_{geg}+\omega_{gl})}\\
\label{eq:tschweb}T_{schweb}=\frac{4\pi}{(\omega_{geg}-\omega_{gl})}.
\end{gather}
\ref{eq:tschnell} und \ref{eq:tschweb} ergibt sich aus der Überlegung, dass sich die Terme mit $(\omega_{geg}+\omega_{gl})$ schnell ändern, während sich die Terme mit $(\omega_{geg}+\omega_{gl})$ deutlich langsamer ändern und so die Schwebung charakterisieren.
\subsubsection{Kopplungsgrad}
Der Kopplungsgrad beschreibt das Verhältnis zwischen der Auslenkung die von außen auf ein Pendel wirkt und der Auslenkung, die das andere Pendel erfährt.
\begin{equation}
k=\frac{x_2}{x_1}=\frac{\omega_{geg}^2-\omega_{gl}^2}{\omega_{geg}^2+\omega_{gl}^2}=\frac{T_{gl}^2-T_{geg}^2}{T_{gl}^2+T_{geg}^2}
\label{eq:kopplung}
\end{equation}
Aus \ref{eq:kopplung} folgt
\begin{equation}
\frac{\Delta \omega}{\omega_0}=\sqrt{\frac{1+k}{1-k}}-1
\label{eq:frequenzspalt}
\end{equation}
Was durch Reihenentwicklungen gleich
\begin{equation}
\frac{\Delta \omega}{\omega_0}= l+\frac{1}{2}k^2+\frac{1}{2}k^3+\dots
\label{eq:3ordnung}
\end{equation}
\section{Versuch: Federpendel}
Ziel des Versuches ist die Bestimmung der Federkonstante $D$ auf zwei verschiedene Arten: Zuerst statisch und anschließend aus der Periodendauer einer Schwingung.

Die Feder ist vertikal vor einem Maßstab aufgehängt. Ein schwarzes Plättchen an der Feder zeigt die Höhe auf dem Maßstab an. Die Feder so wird justiert, dass bei angehängter Waagschale das Plättchen auf 0 zeigt. Drei Gewichte stehen zur Verfügung und werden mit einer Waage gewogen.
Genauso wird die Masse von Feder und Waagschale bestimmt: $m_F=\SI{70.47(1)}{g}$.

Die Gewichte werden nacheinander in die Waagschale gelegt und die Auslenkung abgelesen. Dabei wird ein Spiegel zum parallaxenfreien Ablesen neben die Skala gehalten. Wir erwarten aus \cref{eq:hook} einen linearen Zusammenhang.
\begin{table}[H]
  \centering
  \begin{tabular}{c c c c} \toprule
    Masse $m$ & \SI{49.86(1)}{g} & \SI{99.71(1)}{g} & \SI{199.46(1)}{g} \\
    Auslenkung $x$ & \SI{3.7(2)}{cm} & \SI{7.5(2)}{cm} & \SI{15.2(2)}{cm} \\ 
    Federkonst. $D$ & \SI{13.2(7)}{N/m} & \SI{13.0(4)}{N/m} & \SI{12.9(2)}{N/m} \\ \bottomrule
  \end{tabular}
  \caption{Federkonstante bestimmt aus der Auslenkung}
\label{tab:auslenkung}
\end{table}
Das Ergebnis der statischen Bestimmung ergibt sich also als
\begin{equation}
  D_{\text{stat}}=\SI{13.0(3)}{N/m}
  \label{eq:dstat}
\end{equation}
Nun wird die Feder mit angehängter Masse zusätzlich mit der Hand ausgelenkt und die Periodendauer $50T$ für 50 Schwingungen mit einer Stoppuhr gemessen. Die Federkonstante wird mithilfe von \cref{eq:federkonst} berechnet. Aus derselben Gleichung wird erwartet, dass die Schwingungsdauer mit der Wurzel der Masse steigt.
\begin{table}[H]
  \centering
  \begin{tabular}{c c c c} \toprule
    Masse $m$ & \SI{49.86(1)}{g} & \SI{99.71(1)}{g} & \SI{199.46(1)}{g} \\
    50 Perioden $50T$ & \SI{26.12(50)}{s} & \SI{32.66(50)}{s} & \SI{43.41(50)}{s} \\ 
    Federkonst. $D$ & \SI{10.61(41)}{N/m} & \SI{11.40(35)}{N/m} & \SI{11.68(27)}{N/m} \\ \bottomrule
  \end{tabular}
  \caption{Federkonstante bestimmt aus Schwingungsperioden}
  \label{tab:federschwingung}
\end{table}
Die dynamisch bestimmte Federkonstante lautet also 
\begin{equation}
  D_{\text{dyn}}=\SI{11.23(20)}{N/m}.
  \label{eq:ddyn}
\end{equation}
\section{Versuch: Mathematisches Pendel}
Ziel des Versuches ist die Bestimmung der Erdbeschleunigung $g$ aus der Schwingungsdauer eines Fadenpendels.

Ein Faden ist an einem Ende an einer Halterung befestigt, am anderen Ende hängt eine metallische Kugel unbekannter Masse. Mit einem Maßstab wird die Pendellänge $l$ gemessen und für 3 verschiedene Pendellängen wird das Gewicht mit der Hand ausgelenkt, sodass es zur Schwingung kommt. Die Dauer von 50 Schwingungen wird mit der Stoppuhr gemessen und anschließend die Erdbeschleunigung $g$ über \cref{eq:erdbeschl} berechnet.
\begin{table}[H]
  \centering
  \begin{tabular}{c c c c} \toprule
    Länge $l$ & \SI{33.0(3)}{cm} & \SI{53.0(3)}{cm} & \SI{81.7(3)}{cm} \\
    50 Perioden $50T$ & \SI{57.06(50)}{s} & \SI{72.75(50)}{s} & \SI{90.53(50)}{s} \\
    Erdbeschl. $g$ & \SI{10.0(2)}{m/s^2} & \SI{9.88(15)}{m/s^2} & \SI{9.84(11)}{m/s^2} \\ \bottomrule
  \end{tabular}
  \caption{Messergebnis zum Fadenpendel}
  \label{tab:fadenpendel}
\end{table}
Aus dem Mittelwert ergibt sich:
\begin{equation}
  g=\SI{9.91(09)}{m/s^2}
  \label{eq:gpendel}
\end{equation}
\begin{figure}[H]
  \centering
  \begin{tikzpicture}
    \begin{axis}[
      width=15 cm,
      height=10 cm,
      xmin=5, xmax=10,
      ymin=1, ymax=2,
      xlabel={Wurzel der Pendellänge $\sqrt{l}$ [\si{\sqrt{cm}}]},
      ylabel={Periodendauer $T$ \si{s}}
    ]
    \addplot+[only marks, error bars/.cd, y dir=both, y fixed=0.01, x dir=both, x explicit] table[x error index=2] {fadenpendel.txt};
    \pgfplotsset{cycle list shift=-1};
    \addplot+[mark=none, domain=5:10] {0.199*x};
    \end{axis}
  \end{tikzpicture}
  \caption{Messergebnis zur Erdbeschleunigung}
  \label{fig:fadenpendel}
\end{figure}
\section{Versuch: Gekoppeltes Pendel}

In diesem Versuchsteil wird ein gekoppeltes Pendel betrachtet, welches aus zwei einzelnen Pendeln, die mit einer weichen Feder verbunden sind, besteht. Die gespannte Feder sorgt dafür, dass sich die beiden Pendel gegenseitig beeinflussen. Zur Messung der Perioden des Pendels wurde ein Ultraschall-Entfernungsmesser genutzt, der auf das eine Pendel ausgerichtet war. Dieses war jedoch nicht genau kalibriert, weshalb die Entfernungsmessungen im weiteren Verlauf mit einem groben Fehler behaftet sind. Dies ist jedoch nicht weiter tragisch, da nur die Schwingungsperioden in der weiteren Betrachtung wichtig sind und nicht die Amplituden bzw. ihr Wert.
\subsection{Einzelnes Pendel}
Als erstes wurde die Schwingungsdauer eines einzelnen Pendels, ohne weitere Koppelung, bestimmt. Dabei beträgt die Länge des Pendels $L = (1,572 \pm 0,003) m$.

\begin{figure}[H]
  \centering
  \begin{tikzpicture}
    \begin{axis}[
      width=15 cm,
      height=10 cm,
      xmin=0, xmax=105,
      ymin=-0.4, ymax=0.7,
      xlabel={Schwingungsdauer $T$ [s]},
      ylabel={Amplitude $a$ [m]},
    ]
      \addplot[mark=square, only marks] table {einzelschwingung.txt};
      
    \end{axis}
  \end{tikzpicture}
  \caption{Amplitude des einzelnen Pendels $a$ gegen die Schwingungsdauer $T_0$ }
  \label{fig:einzelschwingung}
\end{figure}
\begin{figure}[H]
  \centering
  \begin{tikzpicture}
    \begin{axis}[
      width=15 cm,
      height=10 cm,
      xmin=0, xmax=10,
      ymin=-0.4, ymax=0.7,
      xlabel={Schwingungsdauer $T$ [s]},
      ylabel={Amplitude $a$ [m]},
    ]
      \addplot[mark=square, only marks] table {einzelschwingung.txt};
      
    \end{axis}
  \end{tikzpicture}
  \caption{Ausschnitt des Graphen \ref{fig:einzelschwingung} für die ersten $10 s$ }
  \label{fig:durchbiegung}
\end{figure}
Die Schwingungsperiode $T_0$ wurde nun grafisch bestimmt.
\begin{equation}
 T_0=(2,5\pm 0,3)s
\end{equation}
\subsection{Gekoppeltes Pendel}
Nun werden die beiden Pendel wie bereits erwähnt mit einer Feder verbunden. Dies geschieht in $l=(0,507 \pm 0,003)m$ Abstand zu den Fixpunkten der Pendel. Zunächst wurde die Messung mit $l'=(1,530 \pm 0,003)m$ durchgeführt, bei diesem Abstand kam es jedoch im weiteren Verlauf zu keiner genau bestimmbaren Schwebung, weshalb im Weiteren mit $l$ statt $l'$ gearbeitet wurde.

Dieser Aufbau wird mit zwei verschiedenen Federn ,Feder 1 (silbern) und Feder 2 (kupfern), durchgeführt und es werden jeweils der Kopplungsgrad statisch und die Schwingungsperiode für eine gleichsinnige, eine gegensinnige und eine schwebende Schwingung bestimmt.
\subsubsection{Statischer Kopplungsgrad}
Bei diesem Versuch wird ein Pendel um $a$ ausgelenkt und die resultierende Auslenkung $a'$ des anderen Pendels bestimmt.
\begin{table}[H]
  \centering
  \begin{tabular}{l | c | c | c}
    \diagbox{Feder}{Auslenkung $a$ [cm]} & $5\pm 0,1$ & $10\pm 0,1$ & $15\pm 0,1$  \\ \hline
    1 & ($0,2 \pm 0,1$)cm & ($0,5 \pm 0,1$)cm & ($0,8 \pm 0,1$)cm \\
    2 & ($0,0 \pm 0,1$)cm & ($0,2 \pm 0,1$)cm & ($0,3 \pm 0,1$)cm 
  \end{tabular}
  \caption{Statische Bestimmung des Kopplungsgrad}
  \label{tab:statkopplung}
\end{table}
Daraus ergibt sich nach \ref{eq:kopplung} ein durchschnittlicher Kopplungsgrad $k$ für die beiden Federn.
\begin{table}[H]
  \centering
  \begin{tabular}{l | c | c }
    Feder & 1 & 2 \\ \hline
    k & $0,047$ & $0,013$
  \end{tabular}
  \caption{Statisch bestimmter Kopplungsgrad}
  \label{tab:statkopplungwert}
\end{table}
\subsubsection{Dynamischer Kopplungsgrad}

\paragraph{Gleichsinnige Schwingung}
Das gekoppelte Pendel wird zu einer gleichsinnigen Schwingung angeregt und die Schwingungsperiode wird über den Ultraschall-Entfernungsmessung bestimmt.
\begin{figure}[H]
  \centering
  \begin{tikzpicture}
    \begin{axis}[
      width=15 cm,
      height=10 cm,
      xmin=0, xmax=50,
      ymin=0, ymax=2,
      xlabel={Schwingungsdauer $T$ [s]},
      ylabel={Amplitude $a$ [m]},
      legend entries={Feder 1,Feder 2}
    ]
      \addplot[mark=square, only marks] table {tgleichsilber2.txt};
      \addplot[mark=*, only marks, ] table {tgleichkupfer.txt};
      
    \end{axis}
  \end{tikzpicture}
  \caption{Amplitude des gekoppelten Pendels $a$ gegen die Schwingungsdauer $T_0$ }
  \label{fig:gleichschwingung}
\end{figure}
\begin{figure}[H]
  \centering
  \begin{tikzpicture}
    \begin{axis}[
      width=15 cm,
      height=10 cm,
      xmin=0, xmax=7.5,
      ymin=0, ymax=2,
      xlabel={Schwingungsdauer $T$ [s]},
      ylabel={Amplitude $a$ [m]},
      legend entries={Feder 1,Feder 2}
    ]
      \addplot[mark=square, only marks] table {tgleichsilber2.txt};
      \addplot[mark=*, only marks, ] table {tgleichkupfer.txt};
      
    \end{axis}
  \end{tikzpicture}
  \caption{Aussschnitt von Abbildung \ref{fig:gleichschwingung} }
  \label{fig:ausschnittgleich}
\end{figure}
Aus Übersichtsgründen wurde die Messung von Feder 2 um $1$ erhöht, was möglich war, da die Amplitude keine Rolle in der Messung spielt.Die Schwingungsperiode $T_{gl}$ wurde nun grafisch bestimmt.
\begin{table}[H]
  \centering
  \begin{tabular}{l | c | c }
    Feder & 1 & 2\\ \hline
    Schwingungsperiode $T_{gl}$ [s]& $2,5\pm 0,1$ & $2,5 \pm 0,1$
    
  \end{tabular}
  \caption{Schwingungsperiode bei gleichsinniger Bewegung}
  \label{tab:tgleich}
\end{table}
\paragraph{Gegensinnige Schwingung}
Das gekoppelte Pendel wird zu einer gegensinnigen Schwingung angeregt und die Schwingungsperiode wird über den Ultraschall-Entfernungsmessung bestimmt.
\begin{figure}[H]
  \centering
  \begin{tikzpicture}
    \begin{axis}[
      width=15 cm,
      height=10 cm,
      xmin=0, xmax=50,
      ymin=0, ymax=2,
      xlabel={Schwingungsdauer $T$ [s]},
      ylabel={Amplitude $a$ [m]},
      legend entries={Feder 1,Feder 2}
    ]
      \addplot[mark=square, only marks] table {tgegensilber2.txt};
      \addplot[mark=*, only marks, ] table {tgegenkupfer.txt};
      
    \end{axis}
  \end{tikzpicture}
  \caption{Amplitude des gekoppelten Pendels $a$ gegen die Schwingungsdauer $T_0$ }
  \label{fig:gegenschwingung}
\end{figure}
\begin{figure}[H]
  \centering
  \begin{tikzpicture}
    \begin{axis}[
      width=15 cm,
      height=10 cm,
      xmin=0, xmax=7.5,
      ymin=0, ymax=2,
      xlabel={Schwingungsdauer $T$ [s]},
      ylabel={Amplitude $a$ [m]},
      legend entries={Feder 1,Feder 2}
    ]
      \addplot[mark=square, only marks] table {tgegensilber2.txt};
      \addplot[mark=*, only marks, ] table {tgegenkupfer.txt};
      
    \end{axis}
  \end{tikzpicture}
  \caption{Amplitude des gekoppelten Pendels $a$ gegen die Schwingungsdauer $T_0$ }
  \label{fig:ausschnittgegen}
\end{figure}
Aus Übersichtsgründen wurde die Messung von Feder 2 um $1$ erhöht, was möglich war, da die Amplitude keine Rolle in der Messung spielt.Die Schwingungsperiode $T_{geg}$ wurde nun grafisch bestimmt.
\begin{table}[H]
  \centering
  \begin{tabular}{l | c | c }
    Feder & 1 & 2\\ \hline
    Schwingungsperiode $T_{geg}$ [s]& $2,3\pm 0,1$ & $2,4 \pm 0,1$
    
  \end{tabular}
  \caption{Schwingungsperiode bei gleichsinniger Bewegung}
  \label{tab:tgegen}
\end{table}
\paragraph{Schwebung}
Das gekoppelte Pendel wird zu einer  schwebenden Bewegung angeregt und die Schwingungsperiode wird über den Ultraschall-Entfernungsmessung bestimmt.
\begin{figure}[H]
  \centering
  \begin{tikzpicture}
    \begin{axis}[
      width=15 cm,
      height=10 cm,
      xmin=0, xmax=100,
      ymin=1, ymax=2,
      xlabel={Schwingungsdauer $T$ [s]},
      ylabel={Amplitude $a$ [m]},
      legend entries={Feder 2}
    ]
      \addplot[mark=*, only marks, ] table {tschwebekupfer2.txt};
      
    \end{axis}
  \end{tikzpicture}
  \caption{Amplitude des gekoppelten Pendels $a$ mit Feder 2 gegen die Schwingungsdauer $T_0$ }
  \label{fig:schwebschwingung1}
\end{figure}
\begin{figure}[H]
  \centering
  \begin{tikzpicture}
    \begin{axis}[
      width=15 cm,
      height=10 cm,
      xmin=0, xmax=100,
      ymin=0, ymax=0.25,
      xlabel={Schwingungsdauer $T$ [s]},
      ylabel={Amplitude $a$ [m]},
      legend entries={Feder 1}
    ]
      \addplot[mark=*, only marks, ] table {tschwebsilber.txt};
      
    \end{axis}
  \end{tikzpicture}
  \caption{Amplitude des gekoppelten Pendels $a$ mit Feder 1 gegen die Schwingungsdauer $T_0$ }
  \label{fig:schwebschwingung2}
\end{figure}

Aus Übersichtsgründen wurde die Messung von Feder 2 um $1$ erhöht, was möglich war, da die Amplitude keine Rolle in der Messung spielt.Die Schwingungsperiode $T_{schweb}$ wurde nun grafisch bestimmt.
\begin{table}[H]
  \centering
  \begin{tabular}{l | c | c }
    Feder & 1 & 2\\ \hline
    Schwingungsperiode $T_{schweb}$ [s]& $82,2\pm 0,1$ & $86,6 \pm 0,1$
    
  \end{tabular}
  \caption{Schwingungsperiode bei Schwebungsbewegung}
  \label{tab:tschweb}
\end{table}
\paragraph{Berechnung von $t_{schweb}$}
Nach \ref{eq:tschweb} gilt
\begin{equation}
T_{schweb}=\frac{4 \pi}{(T_{geg}^{-1}-T_{gl}^{-1})}.
\end{equation}
Dies führt zu
\begin{table}[H]
  \centering
  \begin{tabular}{l | c | c }
    Feder & 1 & 2\\ \hline
    Schwingungsperiode $T_{schweb}$ [s]& $361,28 \pm 5$ & $753,98 \pm 5$
    
  \end{tabular}
  \caption{Berechnete Schwebungsdauer }
  \label{tab:tschwebrech}
\end{table}
\paragraph{Berechnung des Kopplungsgrad}
Mit der Formel \ref{eq:kopplung} wird nun der Kopplungsgrad für beide Federn aus bestimmten Schwingungsperioden bestimmt.
\begin{table}[H]
  \centering
  \begin{tabular}{l | c | c }
    Feder & 1 & 2\\ \hline
    Kopplungsgrad $k$ & $0,083$ & $0,04$
    
  \end{tabular}
  \caption{Kopplungsgrad  der einzelnen Federn }
  \label{tab:kopllungsgrad}
\end{table}
\paragraph{Berechnung der Frequenzaufspaltung}
Wenn man Formel \ref{eq:kopplung} in \ref{eq:frequenzspalt} einsetzt, enthält man
\begin{equation}
\frac{\Delta \omega}{\omega_0}=\sqrt{\frac{1+\frac{T_{gl}^2-T_{geg}^2}{T_{gl}^2+T_{geg}^2}}{1-\frac{T_{gl}^2-T_{geg}^2}{T_{gl}^2+T_{geg}^2}}}.
\end{equation}
Dies ergibt für die einzelnen Federn
\begin{table}[H]
  \centering
  \begin{tabular}{l | c | c }
    Feder & 1 & 2\\ \hline
    Frequenzaufspaltung $\frac{\Delta \omega}{\omega_0}$& $0,087$ & $0,042$
    
  \end{tabular}
  \caption{Frequenzspaltung der einzelnen Federn }
  \label{tab:frequenzspalt}
\end{table}

Alternativ kann die Frequenzaufspaltung mit der Formel \ref{eq:3ordnung} bestimmen. Dies führt für die beiden Federn zu folgenden Ergebnissen.
\begin{table}[H]
  \centering
  \begin{tabular}{l | c | c }
    Feder & 1 & 2\\ \hline
    Frequenzaufspaltung $\frac{\Delta \omega}{\omega_0}$& $0,087$ & $0,042$
    
  \end{tabular}
  \caption{Frequenzspaltung der einzelnen Federn }
  \label{tab:frequenzspaltalt}
\end{table}
\section{Diskussion}
\subsection{Federpendel}
Die Abweichung zwischen der statischen und der dynamischen Methode zur Bestimmung der Federkonstante beträgt
\begin{equation}
  \Delta D=\SI{13.0}{N/m}-\SI{11.23}{N/m}=\SI{1.77}{N/m}
  \label{eq:ddiff}
\end{equation}
Diese Abweichung liegt sogar außerhalb des Fehlers und der Wert für jede der drei dynamischen Messungen mit verschiedenen Massen weicht nach unten verglichen mit dem statischen Wert ab. Dies deutet auf einen systematischen Messfehler hin. Möglicherweise ist die Annahme aus \cref{eq:hookreal}, dass die träge Masse des Systems $m+m_F/3$ beträgt, nicht haltbar. Um eine größere Übereinstimmung zwischen dynamischem und statischem Wert zu erreichen, müsste statt $m_F/3$ folgendes Verhältnis gewählt werden (Beispielrechnung bei der längsten Schwingung):
\begin{equation}
  a=m_F^{-1}\cdot\left(\frac{\SI{13.0}{N/m}\cdot (\SI{43.41}{s}/50)^2}{4\pi^2}-\SI{199.46}{g}\right)=1.445
  \label{eq:korrektur}
\end{equation}
\subsection{Mathematisches Pendel}
Der gemessene Wert für die Erdbeschleunigung beträgt $g=\SI{9.91(09)}{m/s^2}$. Laut dem National Institute of Standards and Technology lautet der Vergleichswert $g_{lit}=\SI{9.807}{m/s^2}$ Die relative Abweichung beträgt zwar nur \SI{1.05}{\percent}, liegt jedoch außerhalb des Fehlers. Auch hier liegt wieder jede der drei Messungen über dem Literaturwert. Es fällt aber auf, dass die Messungen für längere Schwingungsdauern näher am erwarteten Wert liegen. Möglicherweise ist die Reaktionszeit beim Zeitstoppen höher als geschätzt, sodass der Fehler in Wirklichkeit größer ist. 
Eine weitere Fehlerquelle ist, dass der Literaturwert sich auf Meereshöhe bezieht, während der erste Stock des Physikgebäudes auf ca. \SI{80}{m} über dem Meeresspiegel liegt. Dies erklärt jedoch nur einen kleinen Anteil der Abweichung.

\subsection{Gekoppeltes Pendel}
Beim Vergleich der berechneten Schwebungsdauer $T_S$ mit der gemessenen Dauer fällt ein deutlich Unterschied zwischen den Werten auf. So ist für Feder 1 der berechnete Wert fast neun ($8,79$) mal so groß und bei Feder 2 sogar $17,41$ mal so groß. Diese Unterschiede sind so drastisch, dass man davon ausgehen muss, dass im Laufe des Versuchs ein systematischer Fehler begangen wurde. Dies ist jedoch nur mit weiteren Messungen zu bestätigen.

Bei dem Vergleich der beiden Kopplungsgrade für die einzelnen Federn ist auch ein großer Unterschied festzustellen, so sind die dynamisch bestimmten Werte bei beiden Federn mehr als doppelt so groß. Dies ist darauf zurückzuführen, dass bei der statischen Bestimmung zu einer Verformung der Pendelstangen kam, was den Wert besonders bei so kleinen Werte deutlich verfälscht.

Der Vergleich der auf unterschiedlich bestimmten Frequenzaufspaltungen, nach den Werten aus Tabelle \ref{tab:frequenzspalt} und \ref{tab:frequenzspaltalt}, zeigt deutlich, dass die gelieferte Näherung \ref{eq:3ordnung} geeignet ist um die Frequenzaufspaltung zu bestimmen. So unterscheiden sich die beiden Werte nur außerhalb der signifikanten Stellen, man kann also sagen, dass die Werte gleich sind. 
\section{Doppelpendel}
Das Doppelpendel besteht aus einem physikalischen Pendel, an dessen freien Ende ein Gelenk mit einem zweiten physikalischen Pendel angebracht ist. Es wurden folgende Beobachtungen gemacht:
\begin{itemize}
  \item Die Schwingung sieht chaotisch aus, d.h. es scheint unmöglich, durch naives Anschauen vorherzusagen, wie sich die Pendel als nächstes bewegen werden.
  \item Auf längere Zeit nimmt die Schwingungsamplitude ab und das Pendel kommt schließlich zum Stillstand, die Schwingung ist also gedämpft. Dies ist auf Luftreibung und die Reibung im Kugellager zurückzuführen.
  \item Bei kleinen Auslenkungen verhält sich das Pendel weniger chaotisch: wird das obere Pendel leicht ausgelenkt und das untere gar nicht, so schwingt das obere normal, aber das untere schlägt ruckartig an den Amplitudenmaxima des oberen aus.
    Wird dagegen das untere Pendel leicht ausgelenkt und das obere gar nicht, so schwingt das obere ruckartig und das untere normal.
\end{itemize}
