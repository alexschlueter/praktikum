% --- Pakete einbinden
% --- Pakete erweitern LaTeX um zusätzliche Funktionen. Dies ist ein Satz nützlicher Pakete.

% Silbentrennung; Sprache wird durch Option bei \documentclass festgelegt
\usepackage{babel}
% Verwendung der Zeichentabelle T1 (Sonderzeichen etc.)
\usepackage[T1]{fontenc}
% Legt die Zeichenkodierung der Eingabedatei fest, z.B. UTF8
\usepackage[utf8]{inputenc}
% Schriftart
\usepackage{lmodern}

% Einige LaTeX-Bugfixes
\usepackage{fixltx2e}
% Nutzen von +, -, *, / in \setlength u.ä. (z.B. \setlength{\a + 3cm})
\usepackage{calc}
% Wird benötigt, um \ifthenelse zu benutzen
\usepackage{xifthen}
% Optionen für eigene definierte Befehle
\usepackage{xparse}

% Verbessertes Aussehen von Text
\usepackage{microtype}
% Automatische Anführungszeichen
\usepackage[autostyle]{csquotes}
% Wird für Kopf- und Fußzeile benötigt
\usepackage{scrpage2}
% Einfaches Wechseln zwischen unterschiedlichen Zeilenabständen
\usepackage{setspace}
% Optionen für Listen (enumerate, itemize, ...)
\usepackage{enumitem}
% Zusätzliche Optionen für Tabellen (tabular)
\usepackage{array}

% Mathepaket (intlimits: Grenzen über/unter Integralzeichen)
\usepackage[intlimits]{amsmath}
% Mathe-Symbole, \mathbb etc.
\usepackage{amssymb}
% Weitere Mathebefehle
\usepackage{mathtools}
% "Schöne" Brüche im Fließtext
\usepackage{xfrac}
% Ermöglicht die Nutzung von \SI{Zahl}{Einheit} u.a.
\usepackage{siunitx}

% Farben
\usepackage{xcolor}
% Zum flexiblen Einbinden von Grafiken (\includegraphics)
\usepackage{graphicx}
% .tex-Dateien mit \includegraphics einbiden
\usepackage{gincltex}
% Abbildungen im Fließtext
\usepackage{wrapfig}
% Zitieren, Bibliographie
\usepackage[style=verbose, backend=biber]{biblatex}
% Darstellung von Captions
\usepackage[font=small, labelfont=bf, format=plain]{caption}
% Abbildungen nebeneinander
\usepackage{subcaption}

% Verlinkt Textstellen im PDF-Dokument
\usepackage[pdfpagelabels, unicode]{hyperref}
% "Schlaue" Referenzen (nach hyperref laden!)
\usepackage{cleveref}

% plots
\usepackage{gnuplottex}
\usepackage{pgfplots}

% Genauere Position von figures
\usepackage{float}

% Mehrere Zeilen in table
\usepackage{multirow}

% Tables mit diagonal getrennten Zeilen
\usepackage{diagbox}

% Bessere tables
\usepackage{booktabs}

% --- Einstellungen
% -- latex
% größere Kopfzeile (wegen LaTeX-Warnung)
\setlength{\headheight}{1.4\baselineskip}
% 1,5-facher Zeilenabstand
\onehalfspacing

% -- biblatex (Literaturverzeichnis)
\IfFileExists{res/literatur.bib}{
	\addbibresource{res/literatur.bib}
}{}

% -- csquotes
% Anführungszeichen automatisch umwandeln
\MakeOuterQuote{"}

% -- siunitx
\sisetup{
	locale=DE,
	separate-uncertainty,
	input-product=*,
	output-product=\cdot,
	quotient-mode=fraction,
	per-mode=fraction,
	fraction-function=\sfrac
}

% -- hyperref
\hypersetup{
	% Links/Verweise mit Kasten der Dicke 0.5pt versehen
	pdfborder={0 0 0.5}
}

% -- cleveref
\crefformat{footnote}{#2\textsuperscript{#1}#3}

