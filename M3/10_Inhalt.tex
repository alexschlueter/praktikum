\section{Einführung}
Wird ein Festkörper reversibel verformt, so nennt man dies eine \textbf{elastische} Verformung.
Dabei treten im Körper Spannungen auf, die der Verformungskraft entgegenwirken (3. Newtonsches Gesetz).

Das \textbf{Hookesche Gesetz} nimmt einen proportionalen Zusammenhang zwischen in der Stärke der Verformung und der im Körper wirkenden Spannung an.
Im Falle einer Zugkraft hat dieses die Form
\begin{equation}
  \sigma = E\cdot \varepsilon,
  \label{eq:hookzug}
\end{equation}
wobei $\sigma$ elastische Zugspannung, $E$ Elastizitätsmodul und $\varepsilon=\frac{\Delta L}{L}$ relative Längenänderung heißen.
Bei einer Scherkraft lautet es
\begin{equation}
  \tau = G\alpha,
  \label{eq:hookscher}
\end{equation}
wobei $\tau$ die Schubspannung, $G$ der Schubmodul und $\alpha$ der Scherwinkel sind. \\

Neben einer Längenänderung kommt es auch zu einer Verkürung der Querabmessung $R$ (z.B. Radius). Die \textbf{Poissonzahl / Querkontraktionszahl} gibt das Verhältnis der relativen Änderungen an:
\begin{equation}
  \mu=-\frac{\Delta R/R}{\Delta L/L}
  \label{eq:poisson}
\end{equation}
Die Größen $E, \mu, G, K$ sind also Materialkonstanten, die das Verhalten eines Körpers unter äußeren Verformungskräften beschreiben, solange diese elastisch bleiben. \\

Wird ein Körper an einem Ende festgespannt und am anderen, freien Ende eine Kraft $F$ angelegt, so spricht man von einer \textbf{Elastischen Biegung}.
Die maximale Durchbiegung am freien Ende des Stabes ist
\begin{equation}
  h_{\text{max}}=\frac{F}{E I_q}\frac{L^3}{3}
  \label{eq:maxbiegung}
\end{equation}
($L$ Länge des Körpers, $z$ Abstand von festem Endek, $E$ Elastizitätsmodul, $R$ lokaler Krümmungsradius). Dabei ist $I_q$ das Flächenträgheitsmoment, welches vom Querschnitt des Körpers abhängt:
\begin{align}
  I_q &= \int y^2 \, \mathrm{d}A \\
  I_{\text{Kreis}}&=\frac{\pi d^4}{64} \qquad \text{$d$ Durchmesser} \\
  I_{\text{Rechteck}}&=\frac{ab^3}{12} \qquad \text{$a$ senkrechte Kantenlänge, $b$ parallele}
  \label{eq:flaechentraegheit}
\end{align}
\\
Eine \textbf{Elastische Torsion} ist eine Verdrillung eines zylindrischen Stabes (Radius $R$, Länge $L$) um eine Torsionswinkel $\varphi$.
Hängt man einen Körper mit Trägheitsmoment $J$ an den Stab und verdreht diesen um $\varphi$, so kommt es zu einer harmonischen Schwingung.
Dabei wirkt ein Direktionsmomment $D^*$ der Auslenkung entgenen:
\begin{equation}
  D^*=\frac{\pi G R^4}{2L}
  \label{eq:direktionsmoment}
\end{equation}
Aus der Periodendauer $T$ kann der Schubmodul errechnet werden:
\begin{equation}
  G= \frac{8\pi LJ}{R^4T^2}
  \label{eq:schubmod}
\end{equation}

Das zur Rechnung benötigte Trägheitsmoment lautet für eine Scheibe mit Masse $m$ und Radius $R$:
\begin{equation}
  J_{\text{Scheibe}}=\frac{1}{2}mR^2
  \label{eq:j_scheibe}
\end{equation}


Um dieses für eine Hantel zu bestimmen, hilft der \textbf{Steinerschen Satz}. Hat ein Körper mit Masse $m$ ein Trägheitsmoment $J_S$ bez. eines Punktes $S$, der im Abstand $a$ zu einem zweiten Punkt $A$ liegt, so lautet das Trägheitsmoment bez. $A$:
\begin{equation}
  J_A=J_S+a^2m
  \label{eq:steiner}
\end{equation}
Es gilt nun für eine Hantel:
\begin{align}
  J_{\text{Achse}}&=m_1\left( \frac{1}{12}l_1^2+\frac{1}{4}r_1^2 \right) \\
  J_{\text{Scheiben}}&=m_2\left(\frac{1}{12}l_2^2+\frac{1}{4}(r_2^2+r_1^2)\right)
  \label{eq:achsescheibe}
\end{align}
\begin{align}
  J_{\text{Hantel}}&=J_{\text{Achse}}+2\cdot J_{\text{Scheiben}}+2m_2a^2
  \label{eq:hantel}
\end{align}
Wird diese als Torsionspendel genutzt, ergibt sich für die Schwingung folgende Beziehung zwischen Periodendauer und Abstand der Scheiben:
\begin{align}
  T^2&=\frac{4\pi^2}{D^*}(J_{\text{Achse}}+2\cdot J_{\text{Scheiben}}+2m_2a^2) \\
  \implies T&=\frac{2\pi}{\sqrt{D^*}}\sqrt{J_{\text{Achse}}+2\cdot J_{\text{Scheiben}}+2m_2a^2}
  \label{eq:hanteldauer}
\end{align}
\section{Versuch: Elastische Biegung}
Es wurden 4 Metallstäbe zur Verfügung gestellt:
\begin{table}[h]
  \centering
  \begin{tabular}{l | c | c | r}
    \# & Aussehen & Form & gefühltes Gewicht \\ \hline
    1 & Gold & rechteckig & - \\
    2 & Gold & zylindrisch & mittel \\
    3 & Silber, hell & zylindrisch & leicht \\
    4 & Silber, dunkel & zylindrisch & schwer
  \end{tabular}
  \caption{Erste Beobachtungen zu den Metallstäben}
  \label{tab:metall_beob}
\end{table}

Mit einer Mikrometerschraube wurde die Dicke der Stäbe an 5 verschiedenen Stellen je 3 mal gemessen (Messwerte siehe Laborbuch):
\begin{table}[H]
  \centering
  \begin{tabular}{l | c | c | r}
    \# & Mittelwert [mm] & Standardabweichung [mm] & relativer Fehler \\ \hline
    1 flachkant & \num{1.96} & \num{0.007} & \SI{.36}{\percent} \\
    1 hochkant & \num{4.96} & \num{0.008} & \SI{.16}{\percent} \\
    2 & \num{2.93} & \num{0.012} & \SI{0.41}{\percent} \\
    3 & \num{2.94} & \num{0.006} & \SI{0.20}{\percent} \\
    4 & \num{2.95} & \num{0.012} & \SI{.41}{\percent}
  \end{tabular}
  \caption{Dicke der Stäbe}
  \label{tab:stabdicke}
\end{table}
Da der relative Fehler sehr gering ist (jeweils $<\SI{.5}{\percent}$), nehmen wir im folgenden an, dass die Dicke der Stäbe über die gesamte Länge konstant gleich dem Mittelwert ist.

Die Länge der Stäbe wurde mit einem Maßband gemessen (Fehler jeweils $\pm \SI{0.1}{\cm}$):
\begin{table}[H]
  \centering
  \begin{tabular}{l | c | c | r}
    \# & Länge [cm] \\ \hline
    1 & \SI{29.4}{\cm} \\
    2 & \SI{29.5}{\cm} \\
    3 & \SI{29.8}{\cm} \\
    4 & \SI{29.2}{\cm}
  \end{tabular}
  \caption{Länge der Stäbe}
  \label{tab:stablänge}
\end{table}

Weiterhin standen 5 verschiedene Gewichte (\SI{5}{g}, \SI{10}{g}, \SI{20}{g}, \SI{50}{g}, \SI{100}{g}) zur Verfügung. Die Stäbe wurden an einem Ende horizontal zum Tisch fest eingespannt, wobei das freie Ende sich vor einem Spiegel mit vertikaler Längenskala befand. 

Pro Gewicht wurde nun die Durchbiegung des Stabes gemessen: Zuerst wurde vor jeder Messung der Nullpunkt des Spiegels justiert, sodass er sich mittig hinter dem freien Stabende befand. Dann wurde das Gewicht in eine kleine Schaukel gelegt und diese am freien Ende angebracht. Eventuelle Schwingungen des Stabendes wurden abgewartet und schließlich die vertikale Auslenkung des Stabendes relativ zum Nullpunkt abgelesen. Dabei wurde der Parallaxenfehler gering gehalten, indem beim Ablesen die Spiegelung des Stabes mit dem Stab selbst ausgerichtet wurde.

Der rechteckige Stab wurde sowohl hochkant als auch flachkant eingespannt und gemessen. \\

Die Erwartung ist, dass die Stäbe sich bei höherem angehängten Gewicht stärker biegen. Laut Theorie (\cref{eq:maxbiegung}) sollte der Zusammenhang linear sein (denn $F=mg$). Möglicherweise sind anfangs als schwerer empfundenen Stäbe schlechter zu biegen als die leichteren, weil ein Zusammenhang zwischen innerer Stabilität und Dichte bestehen könnte. Außerdem wird bei flachkantigem Einspannen des 1. Stabes eine stärkere Durchbiegung als bei hochkantigen erwartet.

\begin{table}[H]
  \centering
  \begin{tabular}{l | c | c | c | c | r}
    \# & \SI{5}{g} [mm] & \SI{10}{g} [mm] & \SI{20}{g} [mm] & \SI{50}{g} [mm] & \SI{100}{g} [mm] \\ \hline
    1 flachkant & \num{1.0} & \num{3.0} & \num{6.0} & \num{14.0} & \num{27.0} \\
    1 hochkant & \num{0.0} & \num{1.0} & \num{2.0} & \num{3.0} & \num{5.0} \\
    2 & \num{1.0} & \num{2.0} & \num{4.0} & \num{11.0} & \num{22.0} \\
    3 & \num{2.0} & \num{4.0} & \num{7.0} & \num{16.0} & \num{31.0} \\
    4 & \num{1.0} & \num{2.0} & \num{3.0} & \num{5.0} & \num{10.0}
  \end{tabular}
  \caption{Messergebnis zur Durchbiegung}
  \label{tab:durchbiegung}
\end{table}
Der Fehler wird geschätzt auf jeweils \SI{\pm .5}{mm}.
\begin{figure}[H]
  \centering
  \begin{tikzpicture}
    \begin{axis}[
      width=15 cm,
      height=14 cm,
      xmin=0, xmax=105,
      ymin=-1, ymax=35,
      xlabel={angehängte Masse $m$ in \si{g}},
      ylabel={Durchbiegung $h$ \si{mm}},
      legend entries={Stab 1 flachkant, Stab 1 hochkant, Stab 2, Stab 3, Stab 4},
      legend style={legend pos=north west}
    ]
      \addplot+[only marks, error bars/.cd, y dir=both, y fixed=0.5] table {stab1flach.txt};
      \addplot+[only marks, error bars/.cd, y dir=both, y fixed=0.5] table {stab1hoch.txt};
      \addplot+[only marks, error bars/.cd, y dir=both, y fixed=0.5] table {stab2.txt};
      \addplot+[only marks, error bars/.cd, y dir=both, y fixed=0.5] table {stab3.txt};
      \addplot+[only marks, error bars/.cd, y dir=both, y fixed=0.5] table {stab4.txt};
      \pgfplotsset{cycle list shift=-5}
      \addplot+[mark=none, domain=0:100] {0.27*x};
      \addplot+[mark=none, domain=0:100] {0.054*x};
      \addplot+[mark=none, domain=0:100] {0.22*x};
      \addplot+[mark=none, domain=0:100] {0.31*x};
      \addplot+[mark=none, domain=0:100] {0.10*x};
    \end{axis}
  \end{tikzpicture}
  \caption{Messergebnis zur Durchbiegung}
  \label{fig:durchbiegung}
\end{figure}

Wie erwartet scheint die Durchbiegung linear mit dem angehängten Gewicht zu steigen. Stab 1 hat sich flachkant stärker durchgebogen als hochkant. Der leichteste Stab 3 hat sich stärker als der mittelschwere (2) und dieser wiederum stärker als der schwerste Stab 4 verbogen. \\

Mit \emph{gnuplot} werden nach dem \emph{least-squares}-Verfahren die Werte der einzelnen Stäbe gegen die Funktion $f(x)=a\cdot x$ gefittet. Ausgabe:
\begin{table}[H]
  \centering
  \begin{tabular}{l | c | c}
    \# & a [\si{mm/g}]  \\ \hline
    1 flachkant & \num{0.27(1)} \\
    1 hochkant & \num{0.054(5)} \\
    2 & \num{.22(1)} \\
    3 & \num{0.31(1)} \\
    4 & \num{.10(1)}
  \end{tabular}
  \caption{Linearer Fit der Durchbiegung}
  \label{tab:durchbiegungsfit}
\end{table}

\cref{eq:maxbiegung} wird umgestellt zu 
\begin{align}
  h_{\text{max}}&=a\cdot m \implies a = \frac{gL^3}{3EI_q} \\
  \implies E&=\frac{g}{aI_q}\frac{L^3}{3}
  \label{eq:durchbieg_elasti}
\end{align}
und aus der Steigung $a$ der Elastizitätsmodul berechnet. Es wird erwartet, dass für Stab 1 flachkant derselbe Elastizitätsmodul gilt wie für Stab 1 hochkant, da diese eine Materialgröße ist, die unabhängig von der Art der Einspannung sein sollte.
\begin{table}[H]
  \centering
  \begin{tabular}{l | c}
    \# & Elastizitätsmodul [\si{kN/mm^2}] \\ \hline
    1 flachkant & \num{9.9(4)e1} \\
    1 hochkant & \num{7.7(7)e1} \\
    2 & \num{1.1(1)e2} \\
    3 & \num{7.6(3)e1} \\
    4 & \num{2.2(2)e2}
  \end{tabular}
  \caption{Elastizitätsmodul berechnet aus Steigung}
  \label{tab:elastimodul}
\end{table}

Wider Erwarten liegt die Differenz zwischen dem Wert für Stab 1 flachkant / hochkant nicht innherhalb des Fehlerbereiches. Die aufgrund des gefühlten Gewichtes erwartete Reihenfolge der Elastizitätsmodule (Stab 4 > Stab 2 > Stab 3) ist eingetreten.


\section{Versuch: Torsionsschwingungen}
\subsection{Schwingung mit Scheibe}
An einem Draht der Länge $L=1,739 \pm 0,002 m$ und dem Durchmesser $d_{Draht}= 0,49 \pm 0,01 mm$ wird eine Scheibe mit dem Radius $R_{Scheibe}= 7,1 \pm 0,05 cm$ und mit der Masse $m_{Scheibe}=2648g$ (angegebener Wert) angehängt und nach einer Auslenkung von $\phi =180$° wird die Dauer für 3 Schwingungsperioden bestimmt. Der Durchmesser $d_{Draht}$ wurde durch 5 Messungen an verschiedenen Stellen mit einer Mikrometerschraube bestimmt. Aufgrund der geringen Abweichungen wird er über die gesamte Länge des Drahtes als konstant angenommen.
\begin{table}[H]
  \centering
  \begin{tabular}{l | c | c | c | c}
    Messung & 1 & 2& 3 & 4 \\ \hline
    Schwingungsdauer t & $76,5 \pm 0,5 s $ & $91,03\pm 0,5 s $ & $91,44 \pm 0,5 s $ & $91,06\pm 0,5 s $
  \end{tabular}
  \caption{Schwingungsdauer für drei Perioden mit angehängter Scheibe}
  \label{tab:schwingdauerscheibe}
\end{table}
Im weiteren wird Messung $1$ nicht weiter betrachtet, da ihr Wert deutlich von den anderen Werten abweicht und wahrscheinlich ein grober Messfehler vorliegt.

So ergibt sich eine durchschnittliche  Schwingungsdauer von $\overline{t}=\SI{91.18(23)}{s}$.

Zu Bestimmung des Schubmoduls muss man die Formel \ref{eq:j_scheibe} in \ref{eq:schubmod} einsetzen und erhält:

\begin{equation}
G_{Scheibe}=\frac{4\pi LmR_{Scheibe}^2}{R^4T^2}= \frac{4\pi\cdot1,739m\cdot2,648kg\cdot(0,071m)^2}{(0,00049m)^4\frac{91,18s}{3}^2}=\num{5,47d9} \frac{N}{m^2}
\end{equation}
Mit Messfehler lautet der Wert
\begin{equation}
  G_{Scheibe}=\SI{5.47(9)e9}{N/m^2}
  \label{eq:schubmodulergebnis}
\end{equation}
und der relative Fehler ist
\begin{equation}
\frac{\Delta G_{Scheibe}}{G_{Scheibe}}=60,1\%
\end{equation}
\subsection{Schwingung mit Hantel}
Nun wird der Versuchsablauf von ,,Schwingung mit Scheibe'' wiederholt, jedoch wird diesmal anstatt der Scheibe eine Hantel mit verschiebbaren Gewichten an den Seiten an den Draht gehängt.

Die Hantelstange hat eine Masse von $m_{1}=239,05g$ (angegeben), Länge von $l_{1}=25 \pm 0,1 cm$ und einen Radius von $R_{1}=5,99 \pm 0,01mm$.
Die Gewichte sind Zylinder mit einer Masse von $m_{2}=305,2g$ (angegeben), einem Radius von $\overline{R}_{2}=2,5 cm$ und einer Höhe von $\overline{l}_2=19,8mm$.
Durch das Reduzieren der angehängten Masse hat sich der Draht nun auf $\overline{L}=\SI{1.738(2)}{cm}$ verkürzt.
\begin{table}[H]
  \centering
  \begin{tabular}{l | c | c | c }
    Messung & 1 & 2& 3  \\ \hline
    Schwingungsdauer t & $36,47 \pm 0,5 s $ & $36,59\pm 0,5 s $ & $36,66 \pm 0,5 s $ 
  \end{tabular}
  \caption{Schwingungsdauer für drei Perioden mit angehängter Hantelstange ohne Gewichte}
  \label{tab:schwingdauerstange}
\end{table}
So ergibt sich eine durchschnittliche Periode von $t_{Stange}=\SI{12.19(10)}{s}$

Die Gewichte werden nun in 5 verschiedene Positionen auf der Hantelachse eingerastet und der Abstand der Gewichte zum Aufhängepunkt wird jeweils mit einem Lineal gemessen.
\begin{table}[H]
  \centering
  \begin{tabular}{l | c | c | c | c | c}
    Position & 1 & 2& 3 & 4 & 5 \\ \hline
    Abstand zum Aufhängepunkt [cm] & $11,45 \pm 0,1  $ & $9,1 \pm 0,1$ & $6,6 \pm 0,1$ & $4,1 \pm 0,1$ & $1,9 \pm 0,1 $
  \end{tabular}
  \caption{Abstand der einzelnen Massenpositionen zum Aufhängepunkt}
  \label{tab:Abstandgewichte}
\end{table}
Anschließend wird in jeder Konfiguration die Schwingungsdauer für drei Perioden ermittelt:
\begin{table}[H]
  \centering
  \begin{tabular}{l | c | c | c | c | c}
    Position & 1 & 2& 3 & 4 & 5\\ \hline
    Schwingungsdauer $\overline{t}$ & $104,87s $ & $86,56 s $ & $69,57s $ & $54,30s $ & $43,02s$
  \end{tabular}
  \caption{Schwingungsdauer für drei Perioden mit angehängter Hantel mit Gewichten}
  \label{tab:schwingdauerHantel}
\end{table}

\begin{figure}[H]
  \centering
  \begin{tikzpicture}
    \begin{axis}[
      width=15 cm,
      height=10 cm,
      xmin=0, xmax=81000,
      ymin=0, ymax=1250,
      xlabel={Trägheitsfaktor $2ma^2$ [$g\cdot cm^2$]},
      ylabel={Periodenquadrat $T^2$ [$s^2$]}
    ]
      \addplot[mark=square, only marks, error bars/.cd, y dir=both, y fixed=0.5] table {schwingunghantel.txt};
      
    \end{axis}
  \end{tikzpicture}
  \caption{,,Steinerscher Trägheitsfaktor'' $2ma^2$ gegen Periodendauerquadrat aufgetragen}
  \label{fig:schwingungsdauer}
\end{figure}
Die Fehlerbalken in \cref{fig:schwingungsdauer} sind recht kurz und fallen deshalb optisch mit dem Punkt überein.
Wie aus Formel~\cref{eq:hanteldauer} zu erwarten scheint ein linearer Zusammenhang zwischen dem Quadrat des Abstandes ($a^2$) und dem Periodenquadrat $T^2$ zu bestehen.

Mit \emph{gnuplot} werden nach dem \emph{least-squares}-Verfahren die Werte gegen die Funktion $f(x)=m\cdot x+b$ gefittet. Ausgabe:

\begin{table}[H]
  \centering
  \begin{tabular}{l | c | c}
    Variabel & Wert \\ \hline
    m & \num{1.292\pm 0.001e-2} \\
    b & \num{187\pm 6.352} 
  \end{tabular}
  \caption{Linearer Fit}
  \label{tab:durchbiegungsfit}
\end{table}

Wir stellen nun \cref{eq:hanteldauer} um:
\begin{align}
  T^2&=\frac{4\pi^2}{D^*}(J_{\text{Achse}}+2\cdot J_{\text{Scheibe}}+2m_2a^2) \\
  &=m\cdot (2m_2 a^2)+b \\
  \implies D^*&=\frac{4\pi^2}{m} \\
  \implies J_{\text{Scheibe}} &=\frac{1}{2}\left(\frac{D^*}{4 \pi^2}\cdot b - J_{\text{Achse}}\right)
  \label{eq:hantelergebnis}
\end{align}
und aus der Schwingung ohne Scheiben:
\begin{equation}
  J_{\text{Achse}} = \frac{T^2 D^*}{4 \pi^2}
  \label{eq:hantelohnescheibe}
\end{equation}
\begin{table}[H]
  \centering
  \begin{tabular}{l | c | c | c}
    &$D^*$ & $J_{\text{Achse}}$ & $J_{\text{Scheibe}}$ \\ \hline
    aus Fit berechnet & \SI{3.056(1)e-4}{Nm} & \SI{1.15(2)}{gm^2} & \SI{0.149(724)}{gm^2} \\
    aus Abmessungen & - & \SI{1.247(10)}{gm^2} & \SI{0.0604(38)}{gm^2}
  \end{tabular}
  \caption{Ergebnisse zur Hantelschwingung}
  \label{tab:hantelergebnisse}
\end{table}
Als Vergleichswert wurden mit \cref{eq:achsescheibe} die Trägheitsmomente direkt über die Abmessungen berechnet.
\section{Diskussion}
\subsection{Versuch: Elastische Biegung}
Da es sich beim Elastizitätsmodul um eine Materialgröße handelt, sollte diese unabhängig von der Art des Einspannens seien. Dies hat sich in unserer Messung nicht bestätigt:
\begin{equation}
  E_{1,\text{flach}}-E_{1,\text{hoch}} = \SI{2.2(8)e1}{kN/mm^2}
  \label{eq:elastifalsch}
\end{equation}
Grund dafür ist, dass die Methode zur Ablesung der Durchbiegung nur eine signifikante Stelle geliefert hat, denn mit dem Auge sind Längenunterschiede $<\SI{1}{mm}$ schlecht zu erkennen. Dieser Unterschied liegt außerhalb des berecheten Fehlers, weil wie die Unsicherheiten der Eingabedaten für den \emph{gnuplot}-Fit nicht fortpflanzen können (der von gnuplot ausgegebene Fehler bezieht sich nur auf die Ungenauigkeit des Algorithmus, aber die Eingabedaten werden als exakt angenommen).
Sinvoll ausgesagt werden kann, dass der gefühlt leichteste Stab 3 einen deutlich geringeren Elastizitätsmodul hat als der schwerste Stab 4:
\begin{equation}
  \Delta E_{34}=\SI{200}{kN/mm^2} - \SI{80}{kN/mm^2}=\SI{120}{kN/mm^2}
  \label{eq:elastidiff34}
\end{equation}
Laut \footcite{ingenieurwissen} ist der Elastizitätsmodul für verschiedene Werkstoffe:

\begin{table}[H]
  \centering
  \begin{tabular}{l | c}
    Stoff & Elastizitätsmodul [\si{kN/mm^2}] \\ \hline
    Nickellegierungen & 150\ldots 222 \\
    Gusseisen & 66\ldots 172 \\
    Kupfer & 100 \ldots 130 \\
    Bronze & 105 \ldots 124 \\
    Messing & 78 \ldots 123 \\
    Aluminiumlegierungen & 68 \ldots 82 \\
    Magnesiumlegierungen & 42 \ldots 47
  \end{tabular}
  \caption{Literaturwerte für Elastizitätsmodul}
  \label{tab:litwertelasti}
\end{table}

Stab 1 liegt im Bereich von Messing, während Stab 3 im Bereich von Aluminiumlegierungen liegt. Beides deckt sich mit der Farbe. Bei Stab 2 und 4 ist der Fehler zu hoch, als das die Werte sinnvoll verglichen werden könnten, allerdings legt die sehr ähnliche Farbe nahe, dass Stab 2 genauso wie Stab 1 aus Messing besteht, während Gewicht und Aussehen von Stab 4 auf Gusseisen hindeuten.
Dadurch das sämtlichen Stäben ein Material, welches zu den übrigen Erscheinungsmerkmalen passt, zugeordnet werden konnte, kann man davon ausgehen, dass trotz der teilweise ungenauen Messung, Auslenkung eines Stabes, die Messung gelungen ist.
\subsection{Torsionsschwingung mit Scheibe}
Laut \footcite{elektrotechnik} ist der Schubmodul von verschiedenen Werkstoffen:


\begin{table}[H]
  \centering
  \begin{tabular}{l | c}
    Stoff & Schubmodul $G_M$ [$10^4MPa$] \\ \hline
    Magnesium & 1,7 \\
    Aluminium & 2,6 \\
    Titan & 4,5 \\
    Kupfer & 4,6 \\
    Nickel & 7,6 \\
    Stahl & 8,3 \\
    Wolfram & 16,0
  \end{tabular}
  \caption{Literaturwerte für Schubmodul}
  \label{tab:litwertschub}
\end{table}

Leider wurde durch unsere Messung ein Schubmodul von $G=0,547\cdot10^4 MPa$ bestimmt. Dieser Wert liegt außerhalb der angegebenen Tabelle und liegt eher im Bereich von Plastiken. Da der Stab jedoch aus einem glänzendem Metall bestand, ist es anzunehmen, dass im Laufe der Bestimmung eine 10er-Potenz verloren gegangen ist. Der richtige Wert sollte eher bei $G_{soll}=5,47\cdot10^4 MPa$ liegen, was dann für einen Draht aus Nickel oder ähnlichem Metall spräche.

\subsection{Torsionsschwingung mit Hantel}
Bei der Bestimmung der Trägheitsmomente $J_1$ und $J_2$ wurden zwei unterschiedliche Verfahren gewählt, die beim der Bestimmung von $J_1$, dem Trägheitsmoment der Stange, noch vergleichbare Ergebnisse geliefert haben. So liegt $\Delta J_1$ bei nur $0,97gm^2$, dies entspricht gerade mal einem $0,08\%$ relativem Fehler, was ein gutes Ergebnis ist.
Bei der Bestimmung von $J_2$ ist die ablsolute Differenz ähnlich klein, $\Delta J_2=0,089g^2$, dies ist entspricht aber einem relativen Fehler von über $200\%$, deutlich zu groß ist. 
Bei der Bestimmung ist der Methode über das Fitten, die sehr kleinen Werte des Trägheitsmomentes der Hantelscheiben zum Verhängnis geworden.
