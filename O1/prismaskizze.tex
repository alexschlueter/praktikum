\documentclass[ngerman,pstricks,border=12pt]{standalone}
% --- Pakete einbinden
\input{01_PaketeEinstellungen.tex}
\usetikzlibrary{shapes.misc}
\usetikzlibrary{3d}
\begin{document}
\tikzset{
	point/.style={
		thick,
		draw=gray,
		cross out,
		inner sep=0pt,
		minimum width=4pt,
		minimum height=4pt,
	}
}
\begin{tikzpicture}[]
\coordinate (0) at (0,0,0);
\draw (0,0,0) -- node[below] {$a$} (7,0,0) -- (10,0,0) node [right] {Laser};

\draw[canvas is yz plane at x=0] (-4,1) rectangle (4,-1);
\draw[canvas is xy plane at z=1] (8,-1) -- (6,-1) -- (7,1) -- cycle;
\draw[canvas is xy plane at z=-1] (8,-1) -- (6,-1) -- (7,1) -- cycle;
\draw (8,-1,1) -- (8,-1,-1);
\draw (6,-1,1) -- (6,-1,-1);
\draw (7,1,1) -- (7,1,-1);
\node[above] at (0,3,0) {$y_m$};
\node[below] at (0,0,0) {$0$};
\draw (7,0,0) -- (0,3,0);
\draw[dotted] (0,3,0) -- (0);
\node[point] at (0,3,0) {};
\node[point] at (7,0,0) {};
\node[point] at (0,0,0) {};
\draw[canvas is xy plane at z=0] (7,0) ++(180:2cm) arc (180:157:2cm) node[midway, right] {$\delta_m$};
%\draw[decorate, decoration={brace}, xshift=4pt, yshift=4pt] (0,3,0) -- (7,0,0);
\end{tikzpicture}
\end{document}