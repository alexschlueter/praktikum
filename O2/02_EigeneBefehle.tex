% Eigene Befehle eignen sich gut, um Abkürzungen für lange Befehle zu erstellen. Die Syntax ist folgende:
% \newcommand{neuer Befahl}[Anzahl Parameter (optional)]{ein langer Befehl}
% Das folgende Beispiel fügt ein Bild mit bestimmten vorgegebenen Optionen ein:
\newcommand{\centeredImage}[1]{
	\begin{figure}[h!]
		\centering
		\includegraphics[width=0.50\textwidth]{#1}
	\end{figure}
}
% #1 ist dabei ein Parameter, den man \centeredImage übergeben muss. In 10_Titelseite.tex wird dieser Befehl verwendet. Der Parameter ist dort Bilder/titelseite.jpg.
% Benötigt man keine Parameter, dann lässt man [1] weg. Werden zusätzliche Parameter benötigt, dann kann man die Zahl auf maximal 9 erhöhen.

% Ein Befehl, um eine E-Mail-Adresse darzustellen bzw. automatisch zu verlinken
\newcommand{\email}[1]{\href{mailto:#1}{\texttt{#1}}}

\newcommand{\di}{\mathrm{d}}

\makeatletter
\pgfplotsset{
    my filter/.style args={every#1except#2and#3there#4}{%
        /pgfplots/x filter/.append code={%
            \def\myswitch{1}% Switch is by default on, you can do this with a TeX \if too
            \ifnum\coordindex>#2%Ready to turn off the filter?
            % Yes but are we still in the desired interval?
                \ifnum\coordindex<#3
                \def\myswitch{0}% Turn it off
                \fi%
            \fi%No else
            % Now if the switch is on
            \ifnum1=\myswitch%
                \pgfmathsetmacro\temp{int(mod(\coordindex,#1))}%
                    \ifnum0<\temp
                        \let\pgfmathresult\pgfutil@empty
                    \fi%
						\else
								\pgfmathsetmacro\temp{int(mod(\coordindex,#4))}%
                    \ifnum0<\temp
                        \let\pgfmathresult\pgfutil@empty
                    \fi%
            \fi%
        }
    }
}
\makeatother