% Eigene Befehle eignen sich gut, um Abkürzungen für lange Befehle zu erstellen. Die Syntax ist folgende:
% \newcommand{neuer Befahl}[Anzahl Parameter (optional)]{ein langer Befehl}
% Das folgende Beispiel fügt ein Bild mit bestimmten vorgegebenen Optionen ein:
\newcommand{\centeredImage}[1]{
	\begin{figure}[h!]
		\centering
		\includegraphics[width=0.50\textwidth]{#1}
	\end{figure}
}
% #1 ist dabei ein Parameter, den man \centeredImage übergeben muss. In 10_Titelseite.tex wird dieser Befehl verwendet. Der Parameter ist dort Bilder/titelseite.jpg.
% Benötigt man keine Parameter, dann lässt man [1] weg. Werden zusätzliche Parameter benötigt, dann kann man die Zahl auf maximal 9 erhöhen.

% Ein Befehl, um eine E-Mail-Adresse darzustellen bzw. automatisch zu verlinken
\newcommand{\email}[1]{\href{mailto:#1}{\texttt{#1}}}


